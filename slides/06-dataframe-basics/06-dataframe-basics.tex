\documentclass[12pt]{beamer}\usepackage[]{graphicx}\usepackage[]{color}
%% maxwidth is the original width if it is less than linewidth
%% otherwise use linewidth (to make sure the graphics do not exceed the margin)
\makeatletter
\def\maxwidth{ %
  \ifdim\Gin@nat@width>\linewidth
    \linewidth
  \else
    \Gin@nat@width
  \fi
}
\makeatother

\definecolor{fgcolor}{rgb}{0.345, 0.345, 0.345}
\newcommand{\hlnum}[1]{\textcolor[rgb]{0.686,0.059,0.569}{#1}}%
\newcommand{\hlstr}[1]{\textcolor[rgb]{0.192,0.494,0.8}{#1}}%
\newcommand{\hlcom}[1]{\textcolor[rgb]{0.678,0.584,0.686}{\textit{#1}}}%
\newcommand{\hlopt}[1]{\textcolor[rgb]{0,0,0}{#1}}%
\newcommand{\hlstd}[1]{\textcolor[rgb]{0.345,0.345,0.345}{#1}}%
\newcommand{\hlkwa}[1]{\textcolor[rgb]{0.161,0.373,0.58}{\textbf{#1}}}%
\newcommand{\hlkwb}[1]{\textcolor[rgb]{0.69,0.353,0.396}{#1}}%
\newcommand{\hlkwc}[1]{\textcolor[rgb]{0.333,0.667,0.333}{#1}}%
\newcommand{\hlkwd}[1]{\textcolor[rgb]{0.737,0.353,0.396}{\textbf{#1}}}%

\usepackage{framed}
\makeatletter
\newenvironment{kframe}{%
 \def\at@end@of@kframe{}%
 \ifinner\ifhmode%
  \def\at@end@of@kframe{\end{minipage}}%
  \begin{minipage}{\columnwidth}%
 \fi\fi%
 \def\FrameCommand##1{\hskip\@totalleftmargin \hskip-\fboxsep
 \colorbox{shadecolor}{##1}\hskip-\fboxsep
     % There is no \\@totalrightmargin, so:
     \hskip-\linewidth \hskip-\@totalleftmargin \hskip\columnwidth}%
 \MakeFramed {\advance\hsize-\width
   \@totalleftmargin\z@ \linewidth\hsize
   \@setminipage}}%
 {\par\unskip\endMakeFramed%
 \at@end@of@kframe}
\makeatother

\definecolor{shadecolor}{rgb}{.97, .97, .97}
\definecolor{messagecolor}{rgb}{0, 0, 0}
\definecolor{warningcolor}{rgb}{1, 0, 1}
\definecolor{errorcolor}{rgb}{1, 0, 0}
\newenvironment{knitrout}{}{} % an empty environment to be redefined in TeX

\usepackage{alltt}
\usepackage{graphicx}
\usepackage{tikz}
\setbeameroption{hide notes}
\setbeamertemplate{note page}[plain]
\usepackage{listings}

% get rid of junk
\usetheme{default}
\usefonttheme[onlymath]{serif}
\beamertemplatenavigationsymbolsempty
\hypersetup{pdfpagemode=UseNone} % don't show bookmarks on initial view

% named colors
\definecolor{offwhite}{RGB}{255,250,240}
\definecolor{gray}{RGB}{155,155,155}

\ifx\notescolors\undefined % slides

  \definecolor{foreground}{RGB}{80,80,80}
  \definecolor{background}{RGB}{255,255,255}
  \definecolor{title}{RGB}{255,199,0}
  \definecolor{subtitle}{RGB}{89,132,212}
  \definecolor{hilit}{RGB}{248,117,79}
  \definecolor{vhilit}{RGB}{255,111,207}
  \definecolor{lolit}{RGB}{200,200,200}
  \definecolor{lit}{RGB}{255,199,0}
  \definecolor{mdlit}{RGB}{89,132,212}
  \definecolor{link}{RGB}{248,117,79}

\else % notes
  \definecolor{background}{RGB}{255,255,255}
  \definecolor{foreground}{RGB}{24,24,24}
  \definecolor{title}{RGB}{27,94,134}
  \definecolor{subtitle}{RGB}{22,175,124}
  \definecolor{hilit}{RGB}{122,0,128}
  \definecolor{vhilit}{RGB}{255,0,128}
  \definecolor{lolit}{RGB}{95,95,95}
\fi
\definecolor{nhilit}{RGB}{128,0,128}  % hilit color in notes
\definecolor{nvhilit}{RGB}{255,0,128} % vhilit for notes

\newcommand{\hilit}{\color{hilit}}
\newcommand{\vhilit}{\color{vhilit}}
\newcommand{\nhilit}{\color{nhilit}}
\newcommand{\nvhilit}{\color{nvhilit}}
\newcommand{\lit}{\color{lit}}
\newcommand{\mdlit}{\color{mdlit}}
\newcommand{\lolit}{\color{lolit}}

% use those colors
\setbeamercolor{titlelike}{fg=title}
\setbeamercolor{subtitle}{fg=subtitle}
\setbeamercolor{frametitle}{fg=gray}
\setbeamercolor{structure}{fg=subtitle}
\setbeamercolor{institute}{fg=lolit}
\setbeamercolor{normal text}{fg=foreground,bg=background}
%\setbeamercolor{item}{fg=foreground} % color of bullets
%\setbeamercolor{subitem}{fg=hilit}
%\setbeamercolor{itemize/enumerate subbody}{fg=lolit}
\setbeamertemplate{itemize subitem}{{\textendash}}
\setbeamerfont{itemize/enumerate subbody}{size=\footnotesize}
\setbeamerfont{itemize/enumerate subitem}{size=\footnotesize}

% center title of slides
\setbeamertemplate{blocks}[rounded]
\setbeamertemplate{frametitle}[default][center]
% margins
\setbeamersize{text margin left=25pt,text margin right=25pt}

% page number
\setbeamertemplate{footline}{%
    \raisebox{5pt}{\makebox[\paperwidth]{\hfill\makebox[20pt]{\lolit
          \scriptsize\insertframenumber}}}\hspace*{5pt}}

% add a bit of space at the top of the notes page
\addtobeamertemplate{note page}{\setlength{\parskip}{12pt}}

% default link color
\hypersetup{colorlinks, urlcolor={link}}

\ifx\notescolors\undefined % slides
  % set up listing environment
  \lstset{language=bash,
          basicstyle=\ttfamily\scriptsize,
          frame=single,
          commentstyle=,
          backgroundcolor=\color{darkgray},
          showspaces=false,
          showstringspaces=false
          }
\else % notes
  \lstset{language=bash,
          basicstyle=\ttfamily\scriptsize,
          frame=single,
          commentstyle=,
          backgroundcolor=\color{offwhite},
          showspaces=false,
          showstringspaces=false
          }
\fi

% a few macros
\newcommand{\code}[1]{\texttt{#1}}
\newcommand{\hicode}[1]{{\hilit \texttt{#1}}}
\newcommand{\bb}[1]{\begin{block}{#1}}
\newcommand{\eb}{\end{block}}
\newcommand{\bi}{\begin{itemize}}
%\newcommand{\bbi}{\vspace{24pt} \begin{itemize} \itemsep8pt}
\newcommand{\bbi}{\vspace{4pt} \begin{itemize} \itemsep8pt}
\newcommand{\ei}{\end{itemize}}
\newcommand{\bv}{\begin{verbatim}}
\newcommand{\ev}{\end{verbatim}}
\newcommand{\ig}{\includegraphics}
\newcommand{\subt}[1]{{\footnotesize \color{subtitle} {#1}}}
\newcommand{\ttsm}{\tt \small}
\newcommand{\ttfn}{\tt \footnotesize}
\newcommand{\figh}[2]{\centerline{\includegraphics[height=#2\textheight]{#1}}}
\newcommand{\figw}[2]{\centerline{\includegraphics[width=#2\textwidth]{#1}}}



%------------------------------------------------
% end of header
%------------------------------------------------

\title{Basics of Data Frames}
\subtitle{STAT 133}
\author{\href{http://www.gastonsanchez.com}{Gaston Sanchez}}
\institute{Department of Statistics, UC{\textendash}Berkeley}
\date{\href{http://www.gastonsanchez.com}{\tt \scriptsize \color{foreground} gastonsanchez.com}
\\[-4pt]
\href{http://github.com/gastonstat/stat133}{\tt \scriptsize \color{foreground} github.com/gastonstat/stat133}
\\[-4pt]
{\scriptsize Course web: \href{http://www.gastonsanchez.com/stat133}{\tt gastonsanchez.com/stat133}}
}
\IfFileExists{upquote.sty}{\usepackage{upquote}}{}
\begin{document}


{
  \setbeamertemplate{footline}{} % no page number here
  \frame{
    \titlepage
  } 
}

%------------------------------------------------

\begin{frame}
\frametitle{Data Frames}

\bb{Data Frame}
A {\hilit \code{data.frame}} is the primary data structure that R provides for handling tabular data sets (eg spreadsheet like).
\eb

\bb{Function \code{data.frame()}}
The {\hilit \code{data.frame()}} function allows us to create data frames 
\eb

\end{frame}

%------------------------------------------------

\begin{frame}[fragile]
\frametitle{Creating a Data Frame}

\begin{knitrout}\footnotesize
\definecolor{shadecolor}{rgb}{0.969, 0.969, 0.969}\color{fgcolor}\begin{kframe}
\begin{alltt}
\hlcom{# data frame}
\hlstd{df} \hlkwb{<-} \hlkwd{data.frame}\hlstd{(}
  \hlkwc{name} \hlstd{=} \hlkwd{c}\hlstd{(}\hlstr{'Anakin'}\hlstd{,} \hlstr{'Padme'}\hlstd{,} \hlstr{'Luke'}\hlstd{,} \hlstr{'Leia'}\hlstd{),}
  \hlkwc{gender} \hlstd{=} \hlkwd{c}\hlstd{(}\hlstr{'male'}\hlstd{,} \hlstr{'female'}\hlstd{,} \hlstr{'male'}\hlstd{,} \hlstr{'female'}\hlstd{),}
  \hlkwc{height} \hlstd{=} \hlkwd{c}\hlstd{(}\hlnum{1.88}\hlstd{,} \hlnum{1.65}\hlstd{,} \hlnum{1.72}\hlstd{,} \hlnum{1.50}\hlstd{),}
  \hlkwc{weight} \hlstd{=} \hlkwd{c}\hlstd{(}\hlnum{84}\hlstd{,} \hlnum{45}\hlstd{,} \hlnum{77}\hlstd{,} \hlnum{49}\hlstd{)}
\hlstd{)}
\end{alltt}
\end{kframe}
\end{knitrout}

by default, \code{data.frame()} converts strings into factors

\end{frame}

%------------------------------------------------

\begin{frame}[fragile]
\frametitle{Simple data frame \code{df}}

\begin{knitrout}\footnotesize
\definecolor{shadecolor}{rgb}{0.969, 0.969, 0.969}\color{fgcolor}\begin{kframe}
\begin{alltt}
\hlstd{df}
\end{alltt}
\begin{verbatim}
##     name gender height weight
## 1 Anakin   male   1.88     84
## 2  Padme female   1.65     45
## 3   Luke   male   1.72     77
## 4   Leia female   1.50     49
\end{verbatim}
\end{kframe}
\end{knitrout}

\end{frame}

%------------------------------------------------

\begin{frame}
\frametitle{Inspecting Data Frames}
\bi
  \item dimensions (number of rows and columns)
  \item type of data in each column
  \item row names and column names
  \item missing data
  \item overall summary of each variable
\ei

\end{frame}

%------------------------------------------------

\begin{frame}[fragile]
\frametitle{Overall structure}

A summary of the structure can be obtained with {\hilit \code{str()}}

\begin{knitrout}\footnotesize
\definecolor{shadecolor}{rgb}{0.969, 0.969, 0.969}\color{fgcolor}\begin{kframe}
\begin{alltt}
\hlcom{# structure of a data frame}
\hlkwd{str}\hlstd{(df)}
\end{alltt}
\begin{verbatim}
## 'data.frame':	4 obs. of  4 variables:
##  $ name  : Factor w/ 4 levels "Anakin","Leia",..: 1 4 3 2
##  $ gender: Factor w/ 2 levels "female","male": 2 1 2 1
##  $ height: num  1.88 1.65 1.72 1.5
##  $ weight: num  84 45 77 49
\end{verbatim}
\end{kframe}
\end{knitrout}

\end{frame}

%------------------------------------------------

\begin{frame}[fragile]
\frametitle{Function \code{str()}}

\code{str()} applied on data frames provides:
\bi
  \item number of rows
  \item number of variables
  \item name of each column
  \item mode (i.e. type) of each column (e.g. num, int, chr, factor)
  \item number of levels for \code{factor} variables
\ei

\code{str()} is good for visual inspection, but doesn't give you direct access to the displayed information.

\end{frame}

%------------------------------------------------

\begin{frame}[fragile]
\frametitle{Basic Information of Data Frames}

\begin{center}
 \begin{tabular}{l l}
  \hline
   Function & Description \\
  \hline
  \code{dim()} & dimensions (rows and columns) \\  
  \code{nrow()} & number of rows \\
  \code{ncol()} & number of columns \\  
  \code{names()} & name of columns \\
  \code{colnames()} & name of columns  \\
  \code{rownames()} & names of rows \\
  \code{dimnames()} & list with names of rows and columns \\
  \hline
 \end{tabular}
\end{center}

\end{frame}

%------------------------------------------------

\begin{frame}[fragile]
\frametitle{Basic Information of Data Frames}

\begin{knitrout}\footnotesize
\definecolor{shadecolor}{rgb}{0.969, 0.969, 0.969}\color{fgcolor}\begin{kframe}
\begin{alltt}
\hlkwd{dim}\hlstd{(df)} \hlcom{# dimensions in a two element vector}
\end{alltt}
\begin{verbatim}
## [1] 4 4
\end{verbatim}
\begin{alltt}
\hlkwd{nrow}\hlstd{(df)}
\end{alltt}
\begin{verbatim}
## [1] 4
\end{verbatim}
\begin{alltt}
\hlkwd{ncol}\hlstd{(df)}
\end{alltt}
\begin{verbatim}
## [1] 4
\end{verbatim}
\end{kframe}
\end{knitrout}

\end{frame}

%------------------------------------------------

\begin{frame}[fragile]
\frametitle{Basic Information of Data Frames}

\begin{knitrout}\footnotesize
\definecolor{shadecolor}{rgb}{0.969, 0.969, 0.969}\color{fgcolor}\begin{kframe}
\begin{alltt}
\hlkwd{colnames}\hlstd{(df)}
\end{alltt}
\begin{verbatim}
## [1] "name"   "gender" "height" "weight"
\end{verbatim}
\begin{alltt}
\hlkwd{rownames}\hlstd{(df)}
\end{alltt}
\begin{verbatim}
## [1] "1" "2" "3" "4"
\end{verbatim}
\begin{alltt}
\hlkwd{dimnames}\hlstd{(df)} \hlcom{# names in a list}
\end{alltt}
\begin{verbatim}
## [[1]]
## [1] "1" "2" "3" "4"
## 
## [[2]]
## [1] "name"   "gender" "height" "weight"
\end{verbatim}
\end{kframe}
\end{knitrout}

\end{frame}

%------------------------------------------------

\begin{frame}[fragile]
\frametitle{Function \code{object.size()}}

To know how much memory space is allocated for a data frame (or any other R object) we use \code{object.size()}
\begin{knitrout}\footnotesize
\definecolor{shadecolor}{rgb}{0.969, 0.969, 0.969}\color{fgcolor}\begin{kframe}
\begin{alltt}
\hlkwd{object.size}\hlstd{(df)}
\end{alltt}
\begin{verbatim}
## 2136 bytes
\end{verbatim}
\end{kframe}
\end{knitrout}

\end{frame}

%------------------------------------------------

\begin{frame}[fragile]
\frametitle{Functions \code{head()} and \code{tail}}

Inspect the first and last rows, respectively:
\begin{knitrout}\footnotesize
\definecolor{shadecolor}{rgb}{0.969, 0.969, 0.969}\color{fgcolor}\begin{kframe}
\begin{alltt}
\hlcom{# first 3 rows}
\hlkwd{head}\hlstd{(df,} \hlkwc{n} \hlstd{=} \hlnum{3}\hlstd{)}
\end{alltt}
\begin{verbatim}
##     name gender height weight
## 1 Anakin   male   1.88     84
## 2  Padme female   1.65     45
## 3   Luke   male   1.72     77
\end{verbatim}
\begin{alltt}
\hlcom{# last 2 rows}
\hlkwd{tail}\hlstd{(df,} \hlkwc{n} \hlstd{=} \hlnum{2}\hlstd{)}
\end{alltt}
\begin{verbatim}
##   name gender height weight
## 3 Luke   male   1.72     77
## 4 Leia female   1.50     49
\end{verbatim}
\end{kframe}
\end{knitrout}

\end{frame}

%------------------------------------------------

\begin{frame}[fragile]
\frametitle{Function \code{summary()}}

There's also the function \code{summary()}, which provides a descriptive summary of each column
\begin{knitrout}\footnotesize
\definecolor{shadecolor}{rgb}{0.969, 0.969, 0.969}\color{fgcolor}\begin{kframe}
\begin{alltt}
\hlkwd{summary}\hlstd{(df)}
\end{alltt}
\begin{verbatim}
##      name      gender      height          weight     
##  Anakin:1   female:2   Min.   :1.500   Min.   :45.00  
##  Leia  :1   male  :2   1st Qu.:1.613   1st Qu.:48.00  
##  Luke  :1              Median :1.685   Median :63.00  
##  Padme :1              Mean   :1.688   Mean   :63.75  
##                        3rd Qu.:1.760   3rd Qu.:78.75  
##                        Max.   :1.880   Max.   :84.00
\end{verbatim}
\end{kframe}
\end{knitrout}

\end{frame}

%------------------------------------------------

\begin{frame}
\begin{center}
\Huge{\hilit{Elementary Manipulations}}
\end{center}
\end{frame}

%------------------------------------------------

\begin{frame}
\frametitle{Accessing Elements}

There are many different ways in which the elements of a \code{"data.frame"} can be accessed (i.e. retrieved, selected).

\bi
  \item accessing individual cells
  \item accessing sets of cells
  \item accessing entire rows
  \item accessing entire columns
\ei

\end{frame}

%------------------------------------------------

\begin{frame}
\frametitle{Notation System Reminder}

\bb{Notation system to extract values from data frames}
\bbi
  \item to extract values use brackets: \code{[ ]}
  \item inside the brackets specify indices for rows and columns
  \item each index is separated by comma
  \item row indices can be numbers or logicals
  \item column indices can be numbers, logicals, or names
\ei
\eb
\end{frame}

%------------------------------------------------

\begin{frame}[fragile]
\frametitle{Single Cells}

Using row and column indices to access a single cell
\begin{knitrout}\footnotesize
\definecolor{shadecolor}{rgb}{0.969, 0.969, 0.969}\color{fgcolor}\begin{kframe}
\begin{alltt}
\hlcom{# first cell 1,1}
\hlstd{df[}\hlnum{1}\hlstd{,} \hlnum{1}\hlstd{]}
\end{alltt}
\begin{verbatim}
## [1] Anakin
## Levels: Anakin Leia Luke Padme
\end{verbatim}
\begin{alltt}
\hlcom{# cell 3,4 }
\hlstd{df[}\hlnum{3}\hlstd{,} \hlnum{4}\hlstd{]}
\end{alltt}
\begin{verbatim}
## [1] 77
\end{verbatim}
\begin{alltt}
\hlcom{# last cell}
\hlstd{df[}\hlnum{4}\hlstd{,} \hlnum{4}\hlstd{]}
\end{alltt}
\begin{verbatim}
## [1] 49
\end{verbatim}
\end{kframe}
\end{knitrout}

\end{frame}

%------------------------------------------------

\begin{frame}[fragile]
\frametitle{Various Cells}

Using vectors of row and column indices to access various cells
\begin{knitrout}\footnotesize
\definecolor{shadecolor}{rgb}{0.969, 0.969, 0.969}\color{fgcolor}\begin{kframe}
\begin{alltt}
\hlcom{# various adjacent cells}
\hlstd{df[}\hlnum{1}\hlopt{:}\hlnum{3}\hlstd{,} \hlnum{2}\hlopt{:}\hlnum{4}\hlstd{]}
\end{alltt}
\begin{verbatim}
##   gender height weight
## 1   male   1.88     84
## 2 female   1.65     45
## 3   male   1.72     77
\end{verbatim}
\begin{alltt}
\hlcom{# various adjacent cells}
\hlcom{# (permuted order)}
\hlstd{df[}\hlnum{4}\hlopt{:}\hlnum{1}\hlstd{,} \hlnum{3}\hlopt{:}\hlnum{2}\hlstd{]}
\end{alltt}
\begin{verbatim}
##   height gender
## 4   1.50 female
## 3   1.72   male
## 2   1.65 female
## 1   1.88   male
\end{verbatim}
\end{kframe}
\end{knitrout}

\end{frame}

%------------------------------------------------

\begin{frame}[fragile]
\frametitle{Various Cells}

Using vectors of row and column indices to access various cells
\begin{knitrout}\footnotesize
\definecolor{shadecolor}{rgb}{0.969, 0.969, 0.969}\color{fgcolor}\begin{kframe}
\begin{alltt}
\hlcom{# non-adjacent cells}
\hlstd{df[}\hlkwd{c}\hlstd{(}\hlnum{2}\hlstd{,} \hlnum{4}\hlstd{),} \hlkwd{c}\hlstd{(}\hlnum{1}\hlstd{,} \hlnum{3}\hlstd{)]}
\end{alltt}
\begin{verbatim}
##    name height
## 2 Padme   1.65
## 4  Leia   1.50
\end{verbatim}
\end{kframe}
\end{knitrout}

\end{frame}

%------------------------------------------------

\begin{frame}[fragile]
\frametitle{Various Cells}

Using excluding indices
\begin{knitrout}\footnotesize
\definecolor{shadecolor}{rgb}{0.969, 0.969, 0.969}\color{fgcolor}\begin{kframe}
\begin{alltt}
\hlcom{# excluding various adjacent cells}
\hlstd{df[}\hlopt{-}\hlkwd{c}\hlstd{(}\hlnum{1}\hlopt{:}\hlnum{2}\hlstd{),} \hlopt{-}\hlkwd{c}\hlstd{(}\hlnum{2}\hlopt{:}\hlnum{3}\hlstd{)]}
\end{alltt}
\begin{verbatim}
##   name weight
## 3 Luke     77
## 4 Leia     49
\end{verbatim}
\end{kframe}
\end{knitrout}

\end{frame}

%------------------------------------------------

\begin{frame}[fragile]
\frametitle{Retrieving Rows}

Selecting rows
\begin{knitrout}\footnotesize
\definecolor{shadecolor}{rgb}{0.969, 0.969, 0.969}\color{fgcolor}\begin{kframe}
\begin{alltt}
\hlcom{# first row}
\hlstd{df[}\hlnum{1}\hlstd{, ]}
\end{alltt}
\begin{verbatim}
##     name gender height weight
## 1 Anakin   male   1.88     84
\end{verbatim}
\begin{alltt}
\hlcom{# rows 1 to 3}
\hlstd{df[}\hlnum{1}\hlopt{:}\hlnum{3}\hlstd{, ]}
\end{alltt}
\begin{verbatim}
##     name gender height weight
## 1 Anakin   male   1.88     84
## 2  Padme female   1.65     45
## 3   Luke   male   1.72     77
\end{verbatim}
\end{kframe}
\end{knitrout}

\end{frame}

%------------------------------------------------

\begin{frame}[fragile]
\frametitle{Retrieving Rows}

Selecting rows (by excluding indices)
\begin{knitrout}\footnotesize
\definecolor{shadecolor}{rgb}{0.969, 0.969, 0.969}\color{fgcolor}\begin{kframe}
\begin{alltt}
\hlcom{# all rows except first one}
\hlstd{df[}\hlopt{-}\hlnum{1}\hlstd{, ]}
\end{alltt}
\begin{verbatim}
##    name gender height weight
## 2 Padme female   1.65     45
## 3  Luke   male   1.72     77
## 4  Leia female   1.50     49
\end{verbatim}
\begin{alltt}
\hlcom{# rows except 2 to 3}
\hlstd{df[}\hlopt{-}\hlkwd{c}\hlstd{(}\hlnum{2}\hlopt{:}\hlnum{3}\hlstd{), ]}
\end{alltt}
\begin{verbatim}
##     name gender height weight
## 1 Anakin   male   1.88     84
## 4   Leia female   1.50     49
\end{verbatim}
\end{kframe}
\end{knitrout}

\end{frame}

%------------------------------------------------

\begin{frame}[fragile]
\frametitle{Retrieving Columns}

Selecting columns
\begin{knitrout}\footnotesize
\definecolor{shadecolor}{rgb}{0.969, 0.969, 0.969}\color{fgcolor}\begin{kframe}
\begin{alltt}
\hlcom{# 1st column (as a vector)}
\hlstd{df[ ,} \hlnum{1}\hlstd{]}
\end{alltt}
\begin{verbatim}
## [1] Anakin Padme  Luke   Leia  
## Levels: Anakin Leia Luke Padme
\end{verbatim}
\end{kframe}
\end{knitrout}

Using argument \code{drop=FALSE}
\begin{knitrout}\footnotesize
\definecolor{shadecolor}{rgb}{0.969, 0.969, 0.969}\color{fgcolor}\begin{kframe}
\begin{alltt}
\hlcom{# 1st column (as a column)}
\hlstd{df[ ,} \hlnum{1}\hlstd{,} \hlkwc{drop} \hlstd{=} \hlnum{FALSE}\hlstd{]}
\end{alltt}
\begin{verbatim}
##     name
## 1 Anakin
## 2  Padme
## 3   Luke
## 4   Leia
\end{verbatim}
\end{kframe}
\end{knitrout}

\end{frame}

%------------------------------------------------

\begin{frame}[fragile]
\frametitle{Retrieving Columns}

\begin{knitrout}\footnotesize
\definecolor{shadecolor}{rgb}{0.969, 0.969, 0.969}\color{fgcolor}\begin{kframe}
\begin{alltt}
\hlcom{# columns 1 to 3}
\hlstd{df[ ,} \hlnum{1}\hlopt{:}\hlnum{3}\hlstd{]}
\end{alltt}
\begin{verbatim}
##     name gender height
## 1 Anakin   male   1.88
## 2  Padme female   1.65
## 3   Luke   male   1.72
## 4   Leia female   1.50
\end{verbatim}
\begin{alltt}
\hlcom{# columns 2, 4}
\hlstd{df[ ,} \hlkwd{c}\hlstd{(}\hlnum{2}\hlstd{,}\hlnum{4}\hlstd{)]}
\end{alltt}
\begin{verbatim}
##   gender weight
## 1   male     84
## 2 female     45
## 3   male     77
## 4 female     49
\end{verbatim}
\end{kframe}
\end{knitrout}

\end{frame}

%------------------------------------------------

\begin{frame}[fragile]
\frametitle{Retrieving Columns}

Selecting columns by excluding indices
\begin{knitrout}\footnotesize
\definecolor{shadecolor}{rgb}{0.969, 0.969, 0.969}\color{fgcolor}\begin{kframe}
\begin{alltt}
\hlcom{# all columns but 2, 4,}
\hlstd{df[ ,} \hlopt{-}\hlkwd{c}\hlstd{(}\hlnum{2}\hlstd{,}\hlnum{4}\hlstd{)]}
\end{alltt}
\begin{verbatim}
##     name height
## 1 Anakin   1.88
## 2  Padme   1.65
## 3   Luke   1.72
## 4   Leia   1.50
\end{verbatim}
\end{kframe}
\end{knitrout}

\end{frame}

%------------------------------------------------

\begin{frame}
\frametitle{Accessing Single Columns}

Besides using numeric indices, we can also access a single column using its name and following different syntax options:

\bi
  \item \code{df[ ,"name"]}
  \item \code{df[["name"]]}
  \item \code{df\$name} or \code{df\$"name"}
  \item \code{df["name"]}
\ei

\end{frame}

%------------------------------------------------

\begin{frame}[fragile]
\frametitle{Column by Name}

\begin{knitrout}\footnotesize
\definecolor{shadecolor}{rgb}{0.969, 0.969, 0.969}\color{fgcolor}\begin{kframe}
\begin{alltt}
\hlcom{# equivalent ways to retrieve one column}
\hlstd{df[ ,}\hlstr{"name"}\hlstd{]}
\end{alltt}
\begin{verbatim}
## [1] Anakin Padme  Luke   Leia  
## Levels: Anakin Leia Luke Padme
\end{verbatim}
\begin{alltt}
\hlstd{df[[}\hlstr{"gender"}\hlstd{]]}
\end{alltt}
\begin{verbatim}
## [1] male   female male   female
## Levels: female male
\end{verbatim}
\begin{alltt}
\hlstd{df}\hlopt{$}\hlstd{height}
\end{alltt}
\begin{verbatim}
## [1] 1.88 1.65 1.72 1.50
\end{verbatim}
\end{kframe}
\end{knitrout}

\end{frame}

%------------------------------------------------

\begin{frame}[fragile]
\frametitle{Columns by Name}

Retrieve various columns by name:
\begin{knitrout}\footnotesize
\definecolor{shadecolor}{rgb}{0.969, 0.969, 0.969}\color{fgcolor}\begin{kframe}
\begin{alltt}
\hlcom{# vector of names}
\hlstd{df[ ,} \hlkwd{c}\hlstd{(}\hlstr{"name"}\hlstd{,} \hlstr{"gender"}\hlstd{,} \hlstr{"height"}\hlstd{)]}
\end{alltt}
\begin{verbatim}
##     name gender height
## 1 Anakin   male   1.88
## 2  Padme female   1.65
## 3   Luke   male   1.72
## 4   Leia female   1.50
\end{verbatim}
\end{kframe}
\end{knitrout}

\end{frame}

%------------------------------------------------

\begin{frame}
\frametitle{Adding New Elements}

A typical data frame modification consists in adding new elements, that is, new rows and columns.

\end{frame}

%------------------------------------------------

\begin{frame}[fragile]
\frametitle{Adding One Column}

Adding a single column to a data frame 
\begin{knitrout}\footnotesize
\definecolor{shadecolor}{rgb}{0.969, 0.969, 0.969}\color{fgcolor}\begin{kframe}
\begin{alltt}
\hlcom{# adding 'a_vector' as a 'new' column}
\hlstd{df}\hlopt{$}\hlstd{eyecolor} \hlkwb{<-} \hlkwd{c}\hlstd{(}\hlstr{'blue'}\hlstd{,} \hlstr{'brown'}\hlstd{,} \hlstr{'blue'}\hlstd{,} \hlstr{'brown'}\hlstd{)}

\hlstd{df}
\end{alltt}
\begin{verbatim}
##     name gender height weight eyecolor
## 1 Anakin   male   1.88     84     blue
## 2  Padme female   1.65     45    brown
## 3   Luke   male   1.72     77     blue
## 4   Leia female   1.50     49    brown
\end{verbatim}
\end{kframe}
\end{knitrout}

\end{frame}

%------------------------------------------------

\begin{frame}[fragile]
\frametitle{Adding One Column}

Using \textit{column binding} \code{cbind()} to add a column to a data frame 
\begin{knitrout}\footnotesize
\definecolor{shadecolor}{rgb}{0.969, 0.969, 0.969}\color{fgcolor}\begin{kframe}
\begin{alltt}
\hlstd{haircolor} \hlkwb{<-} \hlkwd{c}\hlstd{(}\hlstr{'blond'}\hlstd{,} \hlstr{'brown'}\hlstd{,} \hlstr{'blond'}\hlstd{,} \hlstr{'brown'}\hlstd{)}

\hlcom{# binding a column}
\hlstd{df} \hlkwb{<-} \hlkwd{cbind}\hlstd{(df, haircolor)}

\hlstd{df}
\end{alltt}
\begin{verbatim}
##     name gender height weight eyecolor haircolor
## 1 Anakin   male   1.88     84     blue     blond
## 2  Padme female   1.65     45    brown     brown
## 3   Luke   male   1.72     77     blue     blond
## 4   Leia female   1.50     49    brown     brown
\end{verbatim}
\end{kframe}
\end{knitrout}

\end{frame}

%------------------------------------------------

\begin{frame}[fragile]
\frametitle{Adding One Column}

Remember the recycling rule:
\begin{knitrout}\footnotesize
\definecolor{shadecolor}{rgb}{0.969, 0.969, 0.969}\color{fgcolor}\begin{kframe}
\begin{alltt}
\hlcom{# "human" will be recycled!}
\hlstd{df}\hlopt{$}\hlstd{species} \hlkwb{<-} \hlstr{"human"}

\hlstd{df}
\end{alltt}
\begin{verbatim}
##     name gender height weight eyecolor haircolor species
## 1 Anakin   male   1.88     84     blue     blond   human
## 2  Padme female   1.65     45    brown     brown   human
## 3   Luke   male   1.72     77     blue     blond   human
## 4   Leia female   1.50     49    brown     brown   human
\end{verbatim}
\end{kframe}
\end{knitrout}

\end{frame}

%------------------------------------------------

\begin{frame}[fragile]
\frametitle{Adding Several Columns}

Equivalent ways to add several columns to a data frame
\begin{knitrout}\footnotesize
\definecolor{shadecolor}{rgb}{0.969, 0.969, 0.969}\color{fgcolor}\begin{kframe}
\begin{alltt}
\hlcom{# adding vectors x and y}
\hlstd{df[ ,} \hlkwd{c}\hlstd{(}\hlstr{"x"}\hlstd{,} \hlstr{"y"}\hlstd{)]} \hlkwb{<-} \hlkwd{cbind}\hlstd{(}\hlnum{1}\hlopt{:}\hlnum{4}\hlstd{,} \hlnum{5}\hlopt{:}\hlnum{8}\hlstd{)}

\hlstd{df}
\end{alltt}
\begin{verbatim}
##     name gender height weight eyecolor haircolor species x y
## 1 Anakin   male   1.88     84     blue     blond   human 1 5
## 2  Padme female   1.65     45    brown     brown   human 2 6
## 3   Luke   male   1.72     77     blue     blond   human 3 7
## 4   Leia female   1.50     49    brown     brown   human 4 8
\end{verbatim}
\end{kframe}
\end{knitrout}

\end{frame}

%------------------------------------------------

\begin{frame}[fragile]
\frametitle{Adding Several Columns}

Equivalent ways to add several columns to a data frame
\begin{knitrout}\footnotesize
\definecolor{shadecolor}{rgb}{0.969, 0.969, 0.969}\color{fgcolor}\begin{kframe}
\begin{alltt}
\hlcom{# adding vectors u and v}
\hlstd{uv} \hlkwb{<-} \hlkwd{cbind}\hlstd{(}\hlkwc{u} \hlstd{=} \hlnum{1}\hlopt{:}\hlnum{4}\hlstd{,} \hlkwc{v} \hlstd{=} \hlnum{5}\hlopt{:}\hlnum{8}\hlstd{)}

\hlstd{df} \hlkwb{<-} \hlkwd{cbind}\hlstd{(df, uv)}
\end{alltt}
\end{kframe}
\end{knitrout}

\end{frame}

%------------------------------------------------

\begin{frame}[fragile]
\frametitle{Removing Columns}

Removing columns with the \code{NULL} object
\begin{knitrout}\footnotesize
\definecolor{shadecolor}{rgb}{0.969, 0.969, 0.969}\color{fgcolor}\begin{kframe}
\begin{alltt}
\hlcom{# removing x and y}
\hlstd{df}\hlopt{$}\hlstd{x} \hlkwb{<-} \hlkwa{NULL}
\hlstd{df}\hlopt{$}\hlstd{y} \hlkwb{<-} \hlkwa{NULL}

\hlstd{df}
\end{alltt}
\begin{verbatim}
##     name gender height weight eyecolor haircolor species
## 1 Anakin   male   1.88     84     blue     blond   human
## 2  Padme female   1.65     45    brown     brown   human
## 3   Luke   male   1.72     77     blue     blond   human
## 4   Leia female   1.50     49    brown     brown   human
\end{verbatim}
\end{kframe}
\end{knitrout}

\end{frame}

%------------------------------------------------

\begin{frame}[fragile]
\frametitle{Removing Columns}

Removing columns by reassignment
\begin{knitrout}\footnotesize
\definecolor{shadecolor}{rgb}{0.969, 0.969, 0.969}\color{fgcolor}\begin{kframe}
\begin{alltt}
\hlcom{# removing columns 5, 6, ...}
\hlstd{df} \hlkwb{<-} \hlstd{df[ ,} \hlnum{1}\hlopt{:}\hlnum{4}\hlstd{]}

\hlstd{df}
\end{alltt}
\begin{verbatim}
##     name gender height weight
## 1 Anakin   male   1.88     84
## 2  Padme female   1.65     45
## 3   Luke   male   1.72     77
## 4   Leia female   1.50     49
\end{verbatim}
\end{kframe}
\end{knitrout}

\end{frame}

%------------------------------------------------

\begin{frame}[fragile]
\frametitle{Adding Rows}

\bi
  \item Another operation is adding rows
  \item This can be done with \textit{row binding} \code{rbind()}
  \item When adding rows to a data frame, we need to take into account the mode of each column
  \item If all columns have the same mode, then we can add a vector
  \item If columns have different modes, then we need to add data.frames
\ei

\end{frame}

%------------------------------------------------

\begin{frame}[fragile]
\frametitle{Adding Rows}

Be careful when adding vector rows to data frames!
\begin{knitrout}\footnotesize
\definecolor{shadecolor}{rgb}{0.969, 0.969, 0.969}\color{fgcolor}\begin{kframe}
\begin{alltt}
\hlcom{# new vector}
\hlstd{newone} \hlkwb{<-} \hlkwd{c}\hlstd{(}\hlstr{"Han"}\hlstd{,} \hlstr{'male'}\hlstd{,} \hlnum{1.8}\hlstd{,} \hlnum{80}\hlstd{)}

\hlcom{# trying to add a vector to data frame}
\hlkwd{rbind}\hlstd{(df, newone)}
\end{alltt}


{\ttfamily\noindent\color{warningcolor}{\#\# Warning in `[<-.factor`(`*tmp*`, ri, value = "{}Han"{}): invalid factor level, NA generated}}\begin{verbatim}
##     name gender height weight
## 1 Anakin   male   1.88     84
## 2  Padme female   1.65     45
## 3   Luke   male   1.72     77
## 4   Leia female    1.5     49
## 5   <NA>   male    1.8     80
\end{verbatim}
\end{kframe}
\end{knitrout}

\end{frame}

%------------------------------------------------

\begin{frame}[fragile]
\frametitle{Adding Rows}

Since columns in \code{df} are of different modes, we must create a new row "vector" in \code{data.frame} format
\begin{knitrout}\footnotesize
\definecolor{shadecolor}{rgb}{0.969, 0.969, 0.969}\color{fgcolor}\begin{kframe}
\begin{alltt}
\hlcom{# creating a data frame "vector"}
\hlstd{han} \hlkwb{<-} \hlkwd{data.frame}\hlstd{(}
  \hlkwc{name} \hlstd{=} \hlstr{"Han"}\hlstd{,}
  \hlkwc{gender} \hlstd{=} \hlstr{'male'}\hlstd{,}
  \hlkwc{height} \hlstd{=} \hlnum{1.8}\hlstd{,}
  \hlkwc{weight} \hlstd{=} \hlnum{80}\hlstd{)}

\hlstd{han}
\end{alltt}
\begin{verbatim}
##   name gender height weight
## 1  Han   male    1.8     80
\end{verbatim}
\end{kframe}
\end{knitrout}

\end{frame}

%------------------------------------------------

\begin{frame}[fragile]
\frametitle{Adding Rows}

Use row binding \code{rbind()} to add one or more rows:
\begin{knitrout}\footnotesize
\definecolor{shadecolor}{rgb}{0.969, 0.969, 0.969}\color{fgcolor}\begin{kframe}
\begin{alltt}
\hlcom{# adding 'han' with rbind()}
\hlstd{df} \hlkwb{<-} \hlkwd{rbind}\hlstd{(df, han)}

\hlstd{df}
\end{alltt}
\begin{verbatim}
##     name gender height weight
## 1 Anakin   male   1.88     84
## 2  Padme female   1.65     45
## 3   Luke   male   1.72     77
## 4   Leia female   1.50     49
## 5    Han   male   1.80     80
\end{verbatim}
\end{kframe}
\end{knitrout}

\end{frame}

%------------------------------------------------

\begin{frame}
\frametitle{Arranging Columns}
A less common, but equally important type of \code{data.frame} modification involves rearranging or moving its columns.

\bigskip

The common approach to move columns is to define a vector with the column names in the desired order, and then redefine the current data frame.
\end{frame}

%------------------------------------------------

\begin{frame}[fragile]
\frametitle{Rearranging columns}

\begin{knitrout}\footnotesize
\definecolor{shadecolor}{rgb}{0.969, 0.969, 0.969}\color{fgcolor}\begin{kframe}
\begin{alltt}
\hlcom{# rearranging columns}
\hlstd{df[ ,} \hlkwd{c}\hlstd{(}\hlnum{1}\hlstd{,} \hlnum{4}\hlstd{,} \hlnum{3}\hlstd{,} \hlnum{2}\hlstd{)]}
\end{alltt}
\begin{verbatim}
##     name weight height gender
## 1 Anakin     84   1.88   male
## 2  Padme     45   1.65 female
## 3   Luke     77   1.72   male
## 4   Leia     49   1.50 female
## 5    Han     80   1.80   male
\end{verbatim}
\end{kframe}
\end{knitrout}

\end{frame}

%------------------------------------------------

\begin{frame}[fragile]
\frametitle{Column Names}

Changing column names
\begin{knitrout}\footnotesize
\definecolor{shadecolor}{rgb}{0.969, 0.969, 0.969}\color{fgcolor}\begin{kframe}
\begin{alltt}
\hlcom{# change first column}
\hlkwd{names}\hlstd{(df)[}\hlnum{1}\hlstd{]} \hlkwb{<-} \hlstr{"Name"}

\hlcom{# change weight}
\hlkwd{names}\hlstd{(df)[}\hlnum{4}\hlstd{]} \hlkwb{<-} \hlstr{"wgt"}

\hlstd{df}
\end{alltt}
\begin{verbatim}
##     Name gender height wgt
## 1 Anakin   male   1.88  84
## 2  Padme female   1.65  45
## 3   Luke   male   1.72  77
## 4   Leia female   1.50  49
## 5    Han   male   1.80  80
\end{verbatim}
\end{kframe}
\end{knitrout}

\end{frame}

%------------------------------------------------

\begin{frame}[fragile]
\frametitle{Column Names}

Changing column names
\begin{knitrout}\footnotesize
\definecolor{shadecolor}{rgb}{0.969, 0.969, 0.969}\color{fgcolor}\begin{kframe}
\begin{alltt}
\hlcom{# rename first column}
\hlkwd{names}\hlstd{(df)[}\hlnum{1}\hlstd{]} \hlkwb{<-} \hlstr{"name"}

\hlcom{# rename weight}
\hlkwd{names}\hlstd{(df)[}\hlnum{4}\hlstd{]} \hlkwb{<-} \hlstr{"weight"}

\hlstd{df}
\end{alltt}
\begin{verbatim}
##     name gender height weight
## 1 Anakin   male   1.88     84
## 2  Padme female   1.65     45
## 3   Luke   male   1.72     77
## 4   Leia female   1.50     49
## 5    Han   male   1.80     80
\end{verbatim}
\end{kframe}
\end{knitrout}

\end{frame}

%------------------------------------------------

\begin{frame}[fragile]
\frametitle{Exercise}

Creating new column
\begin{knitrout}\footnotesize
\definecolor{shadecolor}{rgb}{0.969, 0.969, 0.969}\color{fgcolor}\begin{kframe}
\begin{alltt}
\hlcom{# height x weight}
\hlstd{df}\hlopt{$}\hlstd{htwt} \hlkwb{<-} \hlstd{df}\hlopt{$}\hlstd{height} \hlopt{*} \hlstd{df}\hlopt{$}\hlstd{weight}

\hlcom{# gender and species}
\hlstd{df}\hlopt{$}\hlstd{new} \hlkwb{<-} \hlkwd{paste}\hlstd{(df}\hlopt{$}\hlstd{name, df}\hlopt{$}\hlstd{gender,} \hlkwc{sep} \hlstd{=} \hlstr{"_"}\hlstd{)}

\hlstd{df}
\end{alltt}
\begin{verbatim}
##     name gender height weight   htwt          new
## 1 Anakin   male   1.88     84 157.92  Anakin_male
## 2  Padme female   1.65     45  74.25 Padme_female
## 3   Luke   male   1.72     77 132.44    Luke_male
## 4   Leia female   1.50     49  73.50  Leia_female
## 5    Han   male   1.80     80 144.00     Han_male
\end{verbatim}
\end{kframe}
\end{knitrout}

\end{frame}

%------------------------------------------------

\begin{frame}
\begin{center}
\Huge{\hilit{Subsetting}}
\end{center}
\end{frame}

%------------------------------------------------

\begin{frame}[fragile]
\frametitle{Subsetting}

Subsetting using comparisons (logical vectors \code{TRUE, FALSE})
\begin{knitrout}\footnotesize
\definecolor{shadecolor}{rgb}{0.969, 0.969, 0.969}\color{fgcolor}\begin{kframe}
\begin{alltt}
\hlstd{df}\hlopt{$}\hlstd{name} \hlopt{==} \hlstr{"Luke"}
\end{alltt}
\begin{verbatim}
## [1] FALSE FALSE  TRUE FALSE FALSE
\end{verbatim}
\begin{alltt}
\hlcom{# Luke's info}
\hlstd{df[df}\hlopt{$}\hlstd{name} \hlopt{==} \hlstr{"Luke"}\hlstd{, ]}
\end{alltt}
\begin{verbatim}
##   name gender height weight   htwt       new
## 3 Luke   male   1.72     77 132.44 Luke_male
\end{verbatim}
\end{kframe}
\end{knitrout}

\end{frame}

%------------------------------------------------

\begin{frame}[fragile]
\frametitle{Subsetting}

Subsetting using comparisons (logical vectors \code{TRUE, FALSE})
\begin{knitrout}\footnotesize
\definecolor{shadecolor}{rgb}{0.969, 0.969, 0.969}\color{fgcolor}\begin{kframe}
\begin{alltt}
\hlstd{df}\hlopt{$}\hlstd{gender} \hlopt{==} \hlstr{"male"}
\end{alltt}
\begin{verbatim}
## [1]  TRUE FALSE  TRUE FALSE  TRUE
\end{verbatim}
\begin{alltt}
\hlcom{# male subjects}
\hlstd{df[df}\hlopt{$}\hlstd{gender} \hlopt{==} \hlstr{"male"}\hlstd{, ]}
\end{alltt}
\begin{verbatim}
##     name gender height weight   htwt         new
## 1 Anakin   male   1.88     84 157.92 Anakin_male
## 3   Luke   male   1.72     77 132.44   Luke_male
## 5    Han   male   1.80     80 144.00    Han_male
\end{verbatim}
\end{kframe}
\end{knitrout}

\end{frame}

%------------------------------------------------

\begin{frame}[fragile]
\frametitle{Subsetting}

Subsetting with composed statements
\begin{knitrout}\footnotesize
\definecolor{shadecolor}{rgb}{0.969, 0.969, 0.969}\color{fgcolor}\begin{kframe}
\begin{alltt}
\hlcom{# male with height > 1.75}
\hlstd{df[df}\hlopt{$}\hlstd{gender} \hlopt{==} \hlstr{"male"} \hlopt{&} \hlstd{df}\hlopt{$}\hlstd{height} \hlopt{>} \hlnum{1.75}\hlstd{, ]}
\end{alltt}
\begin{verbatim}
##     name gender height weight   htwt         new
## 1 Anakin   male   1.88     84 157.92 Anakin_male
## 5    Han   male   1.80     80 144.00    Han_male
\end{verbatim}
\end{kframe}
\end{knitrout}

\end{frame}

%------------------------------------------------

\begin{frame}[fragile]
\frametitle{Subsetting}

Subsetting statements can become very verbose
\begin{knitrout}\footnotesize
\definecolor{shadecolor}{rgb}{0.969, 0.969, 0.969}\color{fgcolor}\begin{kframe}
\begin{alltt}
\hlcom{# male with height > 1.75 and weight > 80}
\hlstd{df[df}\hlopt{$}\hlstd{gender} \hlopt{==} \hlstr{"male"}
   \hlopt{&} \hlstd{df}\hlopt{$}\hlstd{height} \hlopt{>} \hlnum{1.75}
   \hlopt{&} \hlstd{df}\hlopt{$}\hlstd{weight} \hlopt{>} \hlnum{80}\hlstd{, ]}
\end{alltt}
\begin{verbatim}
##     name gender height weight   htwt         new
## 1 Anakin   male   1.88     84 157.92 Anakin_male
\end{verbatim}
\end{kframe}
\end{knitrout}

\end{frame}

%------------------------------------------------

\begin{frame}[fragile]
\frametitle{Subsetting with \code{subset()}}

To reduce verbose subsetting statements we can use \code{subset()}
\begin{knitrout}\footnotesize
\definecolor{shadecolor}{rgb}{0.969, 0.969, 0.969}\color{fgcolor}\begin{kframe}
\begin{alltt}
\hlcom{# male with height > 1.75}
\hlkwd{subset}\hlstd{(df, gender} \hlopt{==} \hlstr{"male"} \hlopt{&} \hlstd{height} \hlopt{>} \hlnum{1.75}\hlstd{)}
\end{alltt}
\begin{verbatim}
##     name gender height weight   htwt         new
## 1 Anakin   male   1.88     84 157.92 Anakin_male
## 5    Han   male   1.80     80 144.00    Han_male
\end{verbatim}
\begin{alltt}
\hlcom{# male with height > 1.75 and weight > 80}
\hlkwd{subset}\hlstd{(df, gender} \hlopt{==} \hlstr{"male"} \hlopt{&} \hlstd{height} \hlopt{>} \hlnum{1.75} \hlopt{&} \hlstd{weight} \hlopt{>} \hlnum{80}\hlstd{)}
\end{alltt}
\begin{verbatim}
##     name gender height weight   htwt         new
## 1 Anakin   male   1.88     84 157.92 Anakin_male
\end{verbatim}
\end{kframe}
\end{knitrout}

\end{frame}

%------------------------------------------------

\begin{frame}[fragile]
\frametitle{Subsetting with \code{subset()}}

\code{subset()} also allows you to select columns according to a specified condition
\begin{knitrout}\footnotesize
\definecolor{shadecolor}{rgb}{0.969, 0.969, 0.969}\color{fgcolor}\begin{kframe}
\begin{alltt}
\hlcom{# name and weight of male subjects}
\hlkwd{subset}\hlstd{(df,}
       \hlstd{gender} \hlopt{==} \hlstr{"male"}\hlstd{,}
       \hlkwc{select} \hlstd{=} \hlkwd{c}\hlstd{(name, weight))}
\end{alltt}
\begin{verbatim}
##     name weight
## 1 Anakin     84
## 3   Luke     77
## 5    Han     80
\end{verbatim}
\end{kframe}
\end{knitrout}

\end{frame}

%------------------------------------------------

\begin{frame}[fragile]
\frametitle{Subsetting with \code{subset()}}

\code{subset()} also allows you to select columns according to a specified condition
\begin{knitrout}\footnotesize
\definecolor{shadecolor}{rgb}{0.969, 0.969, 0.969}\color{fgcolor}\begin{kframe}
\begin{alltt}
\hlcom{# excluding height of male subjects}
\hlkwd{subset}\hlstd{(df,}
       \hlstd{gender} \hlopt{==} \hlstr{"male"}\hlstd{,}
       \hlkwc{select} \hlstd{=} \hlopt{-}\hlstd{height)}
\end{alltt}
\begin{verbatim}
##     name gender weight   htwt         new
## 1 Anakin   male     84 157.92 Anakin_male
## 3   Luke   male     77 132.44   Luke_male
## 5    Han   male     80 144.00    Han_male
\end{verbatim}
\end{kframe}
\end{knitrout}

\end{frame}

%------------------------------------------------

\begin{frame}
\begin{center}
\Huge{\hilit{Ordering Rows}}
\end{center}
\end{frame}

%------------------------------------------------

\begin{frame}[fragile]
\frametitle{Sorting rows}

Remember \code{sort()} and \code{order()}
\begin{knitrout}\footnotesize
\definecolor{shadecolor}{rgb}{0.969, 0.969, 0.969}\color{fgcolor}\begin{kframe}
\begin{alltt}
\hlcom{# sort() sorts the values}
\hlkwd{sort}\hlstd{(df}\hlopt{$}\hlstd{weight)}
\end{alltt}
\begin{verbatim}
## [1] 45 49 77 80 84
\end{verbatim}
\begin{alltt}
\hlcom{# order() gives you the position}
\hlkwd{order}\hlstd{(df}\hlopt{$}\hlstd{weight)}
\end{alltt}
\begin{verbatim}
## [1] 2 4 3 5 1
\end{verbatim}
\end{kframe}
\end{knitrout}
When sorting rows, we want to work with the ordered positions

\end{frame}

%------------------------------------------------

\begin{frame}[fragile]
\frametitle{Sorting rows}

\begin{knitrout}\footnotesize
\definecolor{shadecolor}{rgb}{0.969, 0.969, 0.969}\color{fgcolor}\begin{kframe}
\begin{alltt}
\hlcom{# sorting rows by weight }
\hlstd{df[}\hlkwd{order}\hlstd{(df}\hlopt{$}\hlstd{weight), ]}
\end{alltt}
\begin{verbatim}
##     name gender height weight   htwt          new
## 2  Padme female   1.65     45  74.25 Padme_female
## 4   Leia female   1.50     49  73.50  Leia_female
## 3   Luke   male   1.72     77 132.44    Luke_male
## 5    Han   male   1.80     80 144.00     Han_male
## 1 Anakin   male   1.88     84 157.92  Anakin_male
\end{verbatim}
\end{kframe}
\end{knitrout}

\end{frame}

%------------------------------------------------

\begin{frame}[fragile]
\frametitle{Sorting rows}

\begin{knitrout}\footnotesize
\definecolor{shadecolor}{rgb}{0.969, 0.969, 0.969}\color{fgcolor}\begin{kframe}
\begin{alltt}
\hlcom{# sorting subjects by height }
\hlstd{df[}\hlkwd{order}\hlstd{(df}\hlopt{$}\hlstd{height),} \hlkwd{c}\hlstd{(}\hlstr{'name'}\hlstd{,} \hlstr{'height'}\hlstd{)]}
\end{alltt}
\begin{verbatim}
##     name height
## 4   Leia   1.50
## 2  Padme   1.65
## 3   Luke   1.72
## 5    Han   1.80
## 1 Anakin   1.88
\end{verbatim}
\end{kframe}
\end{knitrout}

\end{frame}

%------------------------------------------------

\begin{frame}[fragile]
\frametitle{Sorting rows}

\begin{knitrout}\footnotesize
\definecolor{shadecolor}{rgb}{0.969, 0.969, 0.969}\color{fgcolor}\begin{kframe}
\begin{alltt}
\hlcom{# sorting subjects by height in decreasing order}
\hlstd{ht_sort} \hlkwb{<-} \hlkwd{order}\hlstd{(df}\hlopt{$}\hlstd{height,} \hlkwc{decreasing} \hlstd{=} \hlnum{TRUE}\hlstd{)}
\hlstd{df[ht_sort,} \hlkwd{c}\hlstd{(}\hlstr{'name'}\hlstd{,} \hlstr{'height'}\hlstd{)]}
\end{alltt}
\begin{verbatim}
##     name height
## 1 Anakin   1.88
## 5    Han   1.80
## 3   Luke   1.72
## 2  Padme   1.65
## 4   Leia   1.50
\end{verbatim}
\end{kframe}
\end{knitrout}

\end{frame}

%------------------------------------------------

\end{document}

\documentclass[12pt]{beamer}\usepackage[]{graphicx}\usepackage[]{color}
%% maxwidth is the original width if it is less than linewidth
%% otherwise use linewidth (to make sure the graphics do not exceed the margin)
\makeatletter
\def\maxwidth{ %
  \ifdim\Gin@nat@width>\linewidth
    \linewidth
  \else
    \Gin@nat@width
  \fi
}
\makeatother

\definecolor{fgcolor}{rgb}{0.345, 0.345, 0.345}
\newcommand{\hlnum}[1]{\textcolor[rgb]{0.686,0.059,0.569}{#1}}%
\newcommand{\hlstr}[1]{\textcolor[rgb]{0.192,0.494,0.8}{#1}}%
\newcommand{\hlcom}[1]{\textcolor[rgb]{0.678,0.584,0.686}{\textit{#1}}}%
\newcommand{\hlopt}[1]{\textcolor[rgb]{0,0,0}{#1}}%
\newcommand{\hlstd}[1]{\textcolor[rgb]{0.345,0.345,0.345}{#1}}%
\newcommand{\hlkwa}[1]{\textcolor[rgb]{0.161,0.373,0.58}{\textbf{#1}}}%
\newcommand{\hlkwb}[1]{\textcolor[rgb]{0.69,0.353,0.396}{#1}}%
\newcommand{\hlkwc}[1]{\textcolor[rgb]{0.333,0.667,0.333}{#1}}%
\newcommand{\hlkwd}[1]{\textcolor[rgb]{0.737,0.353,0.396}{\textbf{#1}}}%

\usepackage{framed}
\makeatletter
\newenvironment{kframe}{%
 \def\at@end@of@kframe{}%
 \ifinner\ifhmode%
  \def\at@end@of@kframe{\end{minipage}}%
  \begin{minipage}{\columnwidth}%
 \fi\fi%
 \def\FrameCommand##1{\hskip\@totalleftmargin \hskip-\fboxsep
 \colorbox{shadecolor}{##1}\hskip-\fboxsep
     % There is no \\@totalrightmargin, so:
     \hskip-\linewidth \hskip-\@totalleftmargin \hskip\columnwidth}%
 \MakeFramed {\advance\hsize-\width
   \@totalleftmargin\z@ \linewidth\hsize
   \@setminipage}}%
 {\par\unskip\endMakeFramed%
 \at@end@of@kframe}
\makeatother

\definecolor{shadecolor}{rgb}{.97, .97, .97}
\definecolor{messagecolor}{rgb}{0, 0, 0}
\definecolor{warningcolor}{rgb}{1, 0, 1}
\definecolor{errorcolor}{rgb}{1, 0, 0}
\newenvironment{knitrout}{}{} % an empty environment to be redefined in TeX

\usepackage{alltt}
\usepackage{graphicx}
\usepackage{tikz}
\setbeameroption{hide notes}
\setbeamertemplate{note page}[plain]
\usepackage{listings}

% get rid of junk
\usetheme{default}
\usefonttheme[onlymath]{serif}
\beamertemplatenavigationsymbolsempty
\hypersetup{pdfpagemode=UseNone} % don't show bookmarks on initial view

% named colors
\definecolor{offwhite}{RGB}{255,250,240}
\definecolor{gray}{RGB}{155,155,155}

\ifx\notescolors\undefined % slides

  \definecolor{foreground}{RGB}{80,80,80}
  \definecolor{background}{RGB}{255,255,255}
  \definecolor{title}{RGB}{255,199,0}
  \definecolor{subtitle}{RGB}{89,132,212}
  \definecolor{hilit}{RGB}{248,117,79}
  \definecolor{vhilit}{RGB}{255,111,207}
  \definecolor{lolit}{RGB}{200,200,200}
  \definecolor{lit}{RGB}{255,199,0}
  \definecolor{mdlit}{RGB}{89,132,212}
  \definecolor{link}{RGB}{248,117,79}

\else % notes
  \definecolor{background}{RGB}{255,255,255}
  \definecolor{foreground}{RGB}{24,24,24}
  \definecolor{title}{RGB}{27,94,134}
  \definecolor{subtitle}{RGB}{22,175,124}
  \definecolor{hilit}{RGB}{122,0,128}
  \definecolor{vhilit}{RGB}{255,0,128}
  \definecolor{lolit}{RGB}{95,95,95}
\fi
\definecolor{nhilit}{RGB}{128,0,128}  % hilit color in notes
\definecolor{nvhilit}{RGB}{255,0,128} % vhilit for notes

\newcommand{\hilit}{\color{hilit}}
\newcommand{\vhilit}{\color{vhilit}}
\newcommand{\nhilit}{\color{nhilit}}
\newcommand{\nvhilit}{\color{nvhilit}}
\newcommand{\lit}{\color{lit}}
\newcommand{\mdlit}{\color{mdlit}}
\newcommand{\lolit}{\color{lolit}}

% use those colors
\setbeamercolor{titlelike}{fg=title}
\setbeamercolor{subtitle}{fg=subtitle}
\setbeamercolor{frametitle}{fg=gray}
\setbeamercolor{structure}{fg=subtitle}
\setbeamercolor{institute}{fg=lolit}
\setbeamercolor{normal text}{fg=foreground,bg=background}
%\setbeamercolor{item}{fg=foreground} % color of bullets
%\setbeamercolor{subitem}{fg=hilit}
%\setbeamercolor{itemize/enumerate subbody}{fg=lolit}
\setbeamertemplate{itemize subitem}{{\textendash}}
\setbeamerfont{itemize/enumerate subbody}{size=\footnotesize}
\setbeamerfont{itemize/enumerate subitem}{size=\footnotesize}

% center title of slides
\setbeamertemplate{blocks}[rounded]
\setbeamertemplate{frametitle}[default][center]
% margins
\setbeamersize{text margin left=25pt,text margin right=25pt}

% page number
\setbeamertemplate{footline}{%
    \raisebox{5pt}{\makebox[\paperwidth]{\hfill\makebox[20pt]{\lolit
          \scriptsize\insertframenumber}}}\hspace*{5pt}}

% add a bit of space at the top of the notes page
\addtobeamertemplate{note page}{\setlength{\parskip}{12pt}}

% default link color
\hypersetup{colorlinks, urlcolor={link}}

\ifx\notescolors\undefined % slides
  % set up listing environment
  \lstset{language=bash,
          basicstyle=\ttfamily\scriptsize,
          frame=single,
          commentstyle=,
          backgroundcolor=\color{darkgray},
          showspaces=false,
          showstringspaces=false
          }
\else % notes
  \lstset{language=bash,
          basicstyle=\ttfamily\scriptsize,
          frame=single,
          commentstyle=,
          backgroundcolor=\color{offwhite},
          showspaces=false,
          showstringspaces=false
          }
\fi

% a few macros
\newcommand{\code}[1]{\texttt{#1}}
\newcommand{\hicode}[1]{{\hilit \texttt{#1}}}
\newcommand{\bb}[1]{\begin{block}{#1}}
\newcommand{\eb}{\end{block}}
\newcommand{\bi}{\begin{itemize}}
%\newcommand{\bbi}{\vspace{24pt} \begin{itemize} \itemsep8pt}
\newcommand{\bbi}{\vspace{4pt} \begin{itemize} \itemsep8pt}
\newcommand{\ei}{\end{itemize}}
\newcommand{\bv}{\begin{verbatim}}
\newcommand{\ev}{\end{verbatim}}
\newcommand{\ig}{\includegraphics}
\newcommand{\subt}[1]{{\footnotesize \color{subtitle} {#1}}}
\newcommand{\ttsm}{\tt \small}
\newcommand{\ttfn}{\tt \footnotesize}
\newcommand{\figh}[2]{\centerline{\includegraphics[height=#2\textheight]{#1}}}
\newcommand{\figw}[2]{\centerline{\includegraphics[width=#2\textwidth]{#1}}}



%------------------------------------------------
% end of header
%------------------------------------------------

\title{Getting started with ggplot2}
\subtitle{STAT 133}
\author{\href{http://www.gastonsanchez.com}{Gaston Sanchez}}
\institute{Department of Statistics, UC{\textendash}Berkeley}
\date{\href{http://www.gastonsanchez.com}{\tt \scriptsize \color{foreground} gastonsanchez.com}
\\[-4pt]
\href{http://github.com/gastonstat/stat133}{\tt \scriptsize \color{foreground} github.com/gastonstat/stat133}
\\[-4pt]
{\scriptsize Course web: \href{http://www.gastonsanchez.com/stat133}{\tt gastonsanchez.com/stat133}}
}
\IfFileExists{upquote.sty}{\usepackage{upquote}}{}
\begin{document}


{
  \setbeamertemplate{footline}{} % no page number here
  \frame{
    \titlepage
  } 
}

%------------------------------------------------

\begin{frame}
\begin{center}
\Huge{\hilit{ggplot2}}
\end{center}
\end{frame}

%------------------------------------------------

\begin{frame}[fragile]
\frametitle{Resources for \code{"ggplot2"}}

\bbi
  \item Documentation: \url{http://docs.ggplot2.org/} 
  \item Book: \textbf{ggplot2: Elegant Graphics for Data Analysis} (by Hadley Wickham)
  \item Book: \textbf{R Graphics Cookbook} (by Winston Chang)
  \item RStudio ggplot2 cheat sheet \\
  {\tiny \url{https://www.rstudio.com/wp-content/uploads/2015/03/ggplot2-cheatsheet.pdf}}
\ei

\end{frame}

%------------------------------------------------

\begin{frame}[fragile]
\frametitle{package \code{"ggplot2"}}
\begin{knitrout}\footnotesize
\definecolor{shadecolor}{rgb}{0.969, 0.969, 0.969}\color{fgcolor}\begin{kframe}
\begin{alltt}
\hlcom{# remember to install ggplot2 }
\hlcom{# (just once)}
\hlkwd{install.packages}\hlstd{(}\hlstr{"ggplot2"}\hlstd{)}


\hlcom{# load ggplot2}
\hlkwd{library}\hlstd{(ggplot2)}


\hlcom{# see basic documentation}
\hlopt{?}\hlstd{ggplot}
\end{alltt}
\end{kframe}
\end{knitrout}



\end{frame}

%------------------------------------------------

\begin{frame}[fragile]
\frametitle{ggplot2 book}
\begin{center}
\ig[width=4.5cm]{images/ggplot2.png}
\end{center}
\end{frame}

%------------------------------------------------

\begin{frame}[fragile]
\frametitle{R Graphics Cookbook}
\begin{center}
\ig[width=5cm]{images/graphics_cookbook.jpg}
\end{center}
\end{frame}

%------------------------------------------------

\begin{frame}[fragile]
\begin{knitrout}\scriptsize
\definecolor{shadecolor}{rgb}{0.969, 0.969, 0.969}\color{fgcolor}

{\centering \includegraphics[width=.9\linewidth,height=.65\linewidth]{figure/unnamed-chunk-1-1} 

}



\end{knitrout}
\end{frame}

%------------------------------------------------

\begin{frame}[fragile]
\begin{knitrout}\scriptsize
\definecolor{shadecolor}{rgb}{0.969, 0.969, 0.969}\color{fgcolor}

{\centering \includegraphics[width=.9\linewidth,height=.65\linewidth]{figure/unnamed-chunk-2-1} 

}



\end{knitrout}
\end{frame}

%------------------------------------------------

\begin{frame}
\frametitle{About \code{"ggplot2"}}

\bi
  \item \code{"ggplot2"} (by Hadley Wickham) is an R package for producing statistical graphics
  \item It provides a framework based on Leland Wilkinson's \textbf{Grammar of Graphics}
  \item \code{"ggplot2"} provides beautiful plots while taking care of fiddly details like legends, axes, colors, etc.
  \item \code{"ggplot2"} is built on the R graphics package \code{"grid"}
  \item Underlying philosophy is to describe a wide range of graphics with a compact syntax and independent components
\ei

\end{frame}

%------------------------------------------------

\begin{frame}[fragile]
\frametitle{The Grammar of Graphics}
\begin{center}
\ig[width=4.5cm]{images/grammar_graphics.jpg}
\end{center}
\end{frame}

%------------------------------------------------

\begin{frame}
\frametitle{About the Grammar of Graphics}

\bbi
  \item \textit{The Grammar of Graphics} is Wilkinson's attempt to define a theoretical framework for graphics
  \item \textbf{Grammar}: Formal system of rules for generating graphics
  \bi
    \item Some rules are mathematic
    \item Some rules are aesthetic
  \ei
\ei

\end{frame}

%------------------------------------------------

\begin{frame}
\frametitle{About the Grammar of Graphics}

\bb{3 Stages of Graphic Creation}
\bbi
  \item \textbf{Specification}: link data to graphic objects
  \item \textbf{Assembly}: put everything together
  \item \textbf{Display}: render of a graphic
\ei
\eb

\end{frame}

%------------------------------------------------

\begin{frame}
\frametitle{About the Grammar of Graphics}

\bb{Specification}
Link data to graphic objects
\bbi
  \item Data
  \item Transformation of variables (e.g. aggregation)
  \item Scale transformations (e.g. log)
  \item Coordinate system (e.g. cartesian)
  \item Graphic Elements (e.g. points, lines)
  \item Guides (e.g. labels, legends)
\ei
\eb

\end{frame}

%------------------------------------------------

\begin{frame}[fragile]
\frametitle{R package \code{"ggplot2"}}

About \code{"ggplot2"}
\bbi
  \item Default appearance of plots carefully chosen
  \item Designed with visual perception in mind
  \item Inclusion of some components, like legends, are automated
  \item Great flexibility for annotating, editing, and embedding output
\ei

\end{frame}

%------------------------------------------------

\begin{frame}[fragile]
\frametitle{Base graphics -vs- \code{"ggplot2"}}

\begin{columns}[t]
\begin{column}{0.4\textwidth}
\centerline{base graphics}
\begin{knitrout}\footnotesize
\definecolor{shadecolor}{rgb}{0.969, 0.969, 0.969}\color{fgcolor}

{\centering \includegraphics[width=1.05\linewidth,height=1.05\linewidth]{figure/unnamed-chunk-3-1} 

}



\end{knitrout}
\end{column}

\begin{column}{0.5\textwidth}
\centerline{ggplot2}
\begin{knitrout}\footnotesize
\definecolor{shadecolor}{rgb}{0.969, 0.969, 0.969}\color{fgcolor}

{\centering \includegraphics[width=.85\linewidth,height=.8\linewidth]{figure/unnamed-chunk-4-1} 

}



\end{knitrout}
\end{column}
\end{columns}

\end{frame}

%------------------------------------------------

\begin{frame}[fragile]
\frametitle{About \code{"ggplot2"}}

\bbi
  \item \code{"ggplot2"} is the name of the package
  \item The \code{gg} in \code{"ggplot2"} stands for \textit{Grammar of Graphics}
  \item Inspired in the \textbf{Grammar of Graphics} by Lee Wilkinson
  \item \code{"ggplot"} is the class of objects (plots)
  \item \code{ggplot()} is the main function in \code{"ggplot2"}
\ei

\end{frame}

%------------------------------------------------

\begin{frame}
\begin{center}
\Huge{\hilit{What is a Statistical Graphic?}}
\end{center}
\end{frame}

%------------------------------------------------

\begin{frame}[fragile]
\frametitle{Some Data set}
\code{mtcars}
\begin{knitrout}\footnotesize
\definecolor{shadecolor}{rgb}{0.969, 0.969, 0.969}\color{fgcolor}\begin{kframe}
\begin{verbatim}
##                     mpg   hp  cyl
## Mazda RX4          21.0  110    6
## Mazda RX4 Wag      21.0  110    6
## Datsun 710         22.8   93    4
## Hornet 4 Drive     21.4  110    6
## Hornet Sportabout  18.7  175    8
## Valiant            18.1  105    6
## Duster 360         14.3  245    8
## Merc 240D          24.4   62    4
## Merc 230           22.8   95    4
## Merc 280           19.2  123    6
\end{verbatim}
\end{kframe}
\end{knitrout}
\end{frame}

%------------------------------------------------

\begin{frame}[fragile]
\frametitle{What is a statistical graphic?}
\begin{knitrout}\scriptsize
\definecolor{shadecolor}{rgb}{0.969, 0.969, 0.969}\color{fgcolor}

{\centering \includegraphics[width=.95\linewidth,height=.65\linewidth]{figure/unnamed-chunk-6-1} 

}



\end{knitrout}
\end{frame}

%------------------------------------------------

\begin{frame}[fragile]
\frametitle{What is a statistical graphic?}

Elements to draw the chart ``manually'' 
\pause

\bbi
  \item coordinate system
  \item x and y axis (intervals)
  \item axis tick marks
  \item axis labels, and title
  \item points (with colors)
  \item regression line (and ribbon)
  \item legend
\ei

\end{frame}

%------------------------------------------------

\begin{frame}
\frametitle{What is a statistical graphic?}

\bb{Simply put, a statistical graphic is:}
\bi
  \item A mapping from data to aesthetic attributes (color, shape, size) of geometric objects (points, lines, bars)
  \item A plot may also contain statistical transformations of the data
  \item A plot is drawn on a specific coordinate system
  \item Sometimes faceting can be used to get the same plot for different subsets of the dataset
\ei
\eb

\end{frame}

%------------------------------------------------

\begin{frame}
\begin{center}
\Huge{\hilit{Starting with \code{"ggplot2"}}}
\end{center}
\end{frame}

%------------------------------------------------

\begin{frame}[fragile]
\frametitle{\code{starwarstoy.csv}}
\begin{knitrout}\scriptsize
\definecolor{shadecolor}{rgb}{0.969, 0.969, 0.969}\color{fgcolor}\begin{kframe}


{\ttfamily\noindent\color{warningcolor}{\#\# Warning in file(file, "{}rt"{}): cannot open file '/Users/gaston/Documents/stat133/stat133/datasets/starwarstoy.csv': No such file or directory}}

{\ttfamily\noindent\bfseries\color{errorcolor}{\#\# Error in file(file, "{}rt"{}): cannot open the connection}}

{\ttfamily\noindent\bfseries\color{errorcolor}{\#\# Error in eval(expr, envir, enclos): object 'starwars' not found}}\end{kframe}
\end{knitrout}

\end{frame}

%------------------------------------------------

\begin{frame}[fragile]
\frametitle{Scatterplot}
\begin{knitrout}\scriptsize
\definecolor{shadecolor}{rgb}{0.969, 0.969, 0.969}\color{fgcolor}\begin{kframe}


{\ttfamily\noindent\bfseries\color{errorcolor}{\#\# Error in ggplot(data = starwars): object 'starwars' not found}}\end{kframe}
\end{knitrout}
\end{frame}

%------------------------------------------------

\begin{frame}
\frametitle{Main steps in creating ggplot graphics}
\begin{center}
\ig[width=8cm]{images/ggplot_steps.pdf}
\end{center}
\end{frame}

%------------------------------------------------

\begin{frame}[fragile]
\frametitle{Building a scatterplot}

User specifications
\bi
  \item Dataset: \code{starwars} 
  \item Variables: \code{height, weight, jedi}
  \item Geoms: points
  \item Aesthetics (attributes):
  \bi
    \item \textbf{x}: \code{height}
    \item \textbf{y}: \code{weight}
    \item \textbf{color}: \code{jedi}
  \ei
\ei

\end{frame}

%------------------------------------------------

\begin{frame}[fragile]
\frametitle{Scatterplot with \code{"ggplot2"}}
\begin{knitrout}\footnotesize
\definecolor{shadecolor}{rgb}{0.969, 0.969, 0.969}\color{fgcolor}\begin{kframe}
\begin{alltt}
\hlkwd{ggplot}\hlstd{(}\hlkwc{data} \hlstd{= starwars)} \hlopt{+}
  \hlkwd{geom_point}\hlstd{(}\hlkwd{aes}\hlstd{(}\hlkwc{x} \hlstd{= height,} \hlkwc{y} \hlstd{= weight,} \hlkwc{color} \hlstd{= jedi))}
\end{alltt}
\end{kframe}
\end{knitrout}

\pause
\bi
  \item {\hilit \code{ggplot()}} initializes a \code{"ggplot"} object
  \item specify the dataset with {\mdlit \code{data}}
  \item type of geometric object: {\hilit \code{geom\_point()}}
  \item mapping aesthetic attributes to variables with {\hilit \code{aes()}}
  \bi
    \item x-position: \code{height}
    \item y-position: \code{weight}
    \item color: \code{jedi}
  \ei
\ei
\end{frame}

%------------------------------------------------

\begin{frame}[fragile]
\frametitle{Scatterplot with \code{"ggplot2"}}
\begin{knitrout}\scriptsize
\definecolor{shadecolor}{rgb}{0.969, 0.969, 0.969}\color{fgcolor}\begin{kframe}
\begin{alltt}
\hlkwd{ggplot}\hlstd{(}\hlkwc{data} \hlstd{= starwars)} \hlopt{+}
  \hlkwd{geom_point}\hlstd{(}\hlkwd{aes}\hlstd{(}\hlkwc{x} \hlstd{= height,} \hlkwc{y} \hlstd{= weight,} \hlkwc{color} \hlstd{= jedi))}
\end{alltt}


{\ttfamily\noindent\bfseries\color{errorcolor}{\#\# Error in ggplot(data = starwars): object 'starwars' not found}}\end{kframe}
\end{knitrout}
\end{frame}

%------------------------------------------------

\begin{frame}[fragile]
\frametitle{Scatterplot with \code{"ggplot2"}}

Automated things in \code{"ggplot2"}
\bi
  \item Axis labels
  \item Legends (position, labels, symbols)
  \item Choose of colors for points
  \item Background color (e.g. gray)
  \item Grid lines (major and minor)
  \item Axis tick marks
\ei

{\lit \small you can always change the automated elements}

\end{frame}

%------------------------------------------------

\begin{frame}
\frametitle{\code{"ggplot2"} graphics}

\bb{Philosophy of \code{"ggplot2"}}
A graphic is a {\hilit \textbf{mapping}} from \textbf{data} to \textbf{aesthetic attributes} (color, shape, size) of \textbf{geometric objects} (points, lines, bars)
\eb

\end{frame}

%------------------------------------------------

\begin{frame}[fragile]
\frametitle{Scatterplot with \code{"ggplot2"}}
\begin{knitrout}\scriptsize
\definecolor{shadecolor}{rgb}{0.969, 0.969, 0.969}\color{fgcolor}\begin{kframe}
\begin{alltt}
\hlkwd{ggplot}\hlstd{(}\hlkwc{data} \hlstd{= starwars)} \hlopt{+}
  \hlkwd{geom_point}\hlstd{(}\hlkwd{aes}\hlstd{(}\hlkwc{x} \hlstd{= height,} \hlkwc{y} \hlstd{= weight,} \hlkwc{color} \hlstd{= jedi))}
\end{alltt}


{\ttfamily\noindent\bfseries\color{errorcolor}{\#\# Error in ggplot(data = starwars): object 'starwars' not found}}\end{kframe}
\end{knitrout}
\end{frame}

%------------------------------------------------

\begin{frame}
\frametitle{Mapping}
\begin{center}
\ig[width=11cm]{images/ggplot_mapping.pdf}
\end{center}
\end{frame}

%------------------------------------------------

\begin{frame}
\frametitle{\code{"ggplot2"} graphics}

\bb{Philosophy of \code{"ggplot2"}}
A graphic is a {\hilit \textbf{mapping}} from \textbf{data} to \textbf{aesthetic attributes} (color, shape, size) of \textbf{geometric objects} (points, lines, bars)
\eb

\bi
  \item \code{ggplot(data, ...)}
  \item \code{aes()}
  \item \code{geom\_objects()}
\ei

\end{frame}

%------------------------------------------------

\begin{frame}[fragile]
\frametitle{Scatterplot with \code{"ggplot2"}}

How does \code{"ggplot2"} work?
\bi
  \item plots are created piece-by-piece
  \item plot components added with {\hilit \textbf{+}} operator
  \item aesthetic attributes mapped to data values
  \item computation of scales for aesthetic attributes
\ei

\end{frame}

%------------------------------------------------

\begin{frame}[fragile]
\frametitle{How does it work?}
Usually, we specify the data and variables inside the function {\hilit \code{ggplot()}} 
\begin{knitrout}\footnotesize
\definecolor{shadecolor}{rgb}{0.969, 0.969, 0.969}\color{fgcolor}\begin{kframe}
\begin{alltt}
\hlkwd{ggplot}\hlstd{(}\hlkwc{data} \hlstd{= mtcars,} \hlkwd{aes}\hlstd{(}\hlkwc{x} \hlstd{= mpg,} \hlkwc{y} \hlstd{= hp))}
\end{alltt}
\end{kframe}
\end{knitrout}
Note the use of the internal function {\hilit \code{aes()}} to \textit{map} \code{x} to \code{mpg}, and \code{y} to \code{hp}.

\bigskip
Then we \textbf{add a layer} of geometric objects: points in this case
\begin{knitrout}\footnotesize
\definecolor{shadecolor}{rgb}{0.969, 0.969, 0.969}\color{fgcolor}\begin{kframe}
\begin{alltt}
\hlopt{+} \hlkwd{geom_point}\hlstd{()}
\end{alltt}
\end{kframe}
\end{knitrout}
\end{frame}

%------------------------------------------------

\begin{frame}[fragile]
\frametitle{Some alternative options}
\begin{knitrout}\footnotesize
\definecolor{shadecolor}{rgb}{0.969, 0.969, 0.969}\color{fgcolor}\begin{kframe}
\begin{alltt}
\hlcom{# option A}
\hlkwd{ggplot}\hlstd{(}\hlkwc{data} \hlstd{= starwars,}
       \hlkwd{aes}\hlstd{(}\hlkwc{x} \hlstd{= height,} \hlkwc{y} \hlstd{= weight,} \hlkwc{color} \hlstd{= jedi))} \hlopt{+}
  \hlkwd{geom_point}\hlstd{()}
\end{alltt}
\end{kframe}
\end{knitrout}

\pause
\begin{knitrout}\footnotesize
\definecolor{shadecolor}{rgb}{0.969, 0.969, 0.969}\color{fgcolor}\begin{kframe}
\begin{alltt}
\hlcom{# option B}
\hlkwd{ggplot}\hlstd{(}\hlkwc{data} \hlstd{= starwars)} \hlopt{+}
  \hlkwd{geom_point}\hlstd{(}\hlkwd{aes}\hlstd{(}\hlkwc{x} \hlstd{= height,} \hlkwc{y} \hlstd{= weight,} \hlkwc{color} \hlstd{= jedi))}
\end{alltt}
\end{kframe}
\end{knitrout}

\pause
\begin{knitrout}\footnotesize
\definecolor{shadecolor}{rgb}{0.969, 0.969, 0.969}\color{fgcolor}\begin{kframe}
\begin{alltt}
\hlcom{# option C}
\hlkwd{ggplot}\hlstd{()} \hlopt{+}
  \hlkwd{geom_point}\hlstd{(}\hlkwc{data} \hlstd{= starwars,}
             \hlkwd{aes}\hlstd{(}\hlkwc{x} \hlstd{= height,} \hlkwc{y} \hlstd{= weight,} \hlkwc{color} \hlstd{= jedi))}
\end{alltt}
\end{kframe}
\end{knitrout}
\end{frame}

%------------------------------------------------

\begin{frame}
\frametitle{Main inquiries}

\bb{Always ask yourself ...}
\bbi
  \item What is the \textbf{data} set of interest?
  \item What \textbf{variables} will be used to make the plot?
  \item What \textbf{graphics shapes} will be used to display?
  \item What \textbf{features} of the shapes will be used to represent the data values?
\ei
\eb

\end{frame}

%------------------------------------------------

\begin{frame}
\frametitle{\code{"ggplot2"} basics}

\bbi
  \item The data must be in a \code{data.frame}
  \item Variables are mapped to aesthetic attributes
  \item Aesthetic attributes belong to geometric objects \textbf{geoms} (points, lines, polygons)
\ei

\end{frame}

%------------------------------------------------

\begin{frame}
\frametitle{Basic Terminology}

\bbi
  \item \textbf{ggplot()} - The main function where you specify the dataset and variables to plot
  \item \textbf{geoms} - geometric objetcs
  \bi
    \item \code{geom\_point()}, \code{geom\_bar()}, \code{geom\_line(), \code{geom\_density()}}
  \ei
  \item \textbf{aes} - aesthetics (i.e. attributes)
  \bi
    \item shape, color, fill, linetype
  \ei
\ei

\end{frame}

%------------------------------------------------

\begin{frame}
\frametitle{Warning}

\code{"ggplot2"} comes with the function {\hilit \code{qplot()}} (i.e. \textit{quick plot}). Avoid using it! 

\bigskip
As Karthik Ram says: ``you'll end up unlearning and relearning a good bit''
\end{frame}

%------------------------------------------------

\end{document}

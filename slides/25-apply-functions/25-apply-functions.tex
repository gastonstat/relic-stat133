\documentclass[12pt]{beamer}\usepackage[]{graphicx}\usepackage[]{color}
%% maxwidth is the original width if it is less than linewidth
%% otherwise use linewidth (to make sure the graphics do not exceed the margin)
\makeatletter
\def\maxwidth{ %
  \ifdim\Gin@nat@width>\linewidth
    \linewidth
  \else
    \Gin@nat@width
  \fi
}
\makeatother

\definecolor{fgcolor}{rgb}{0.345, 0.345, 0.345}
\newcommand{\hlnum}[1]{\textcolor[rgb]{0.686,0.059,0.569}{#1}}%
\newcommand{\hlstr}[1]{\textcolor[rgb]{0.192,0.494,0.8}{#1}}%
\newcommand{\hlcom}[1]{\textcolor[rgb]{0.678,0.584,0.686}{\textit{#1}}}%
\newcommand{\hlopt}[1]{\textcolor[rgb]{0,0,0}{#1}}%
\newcommand{\hlstd}[1]{\textcolor[rgb]{0.345,0.345,0.345}{#1}}%
\newcommand{\hlkwa}[1]{\textcolor[rgb]{0.161,0.373,0.58}{\textbf{#1}}}%
\newcommand{\hlkwb}[1]{\textcolor[rgb]{0.69,0.353,0.396}{#1}}%
\newcommand{\hlkwc}[1]{\textcolor[rgb]{0.333,0.667,0.333}{#1}}%
\newcommand{\hlkwd}[1]{\textcolor[rgb]{0.737,0.353,0.396}{\textbf{#1}}}%

\usepackage{framed}
\makeatletter
\newenvironment{kframe}{%
 \def\at@end@of@kframe{}%
 \ifinner\ifhmode%
  \def\at@end@of@kframe{\end{minipage}}%
  \begin{minipage}{\columnwidth}%
 \fi\fi%
 \def\FrameCommand##1{\hskip\@totalleftmargin \hskip-\fboxsep
 \colorbox{shadecolor}{##1}\hskip-\fboxsep
     % There is no \\@totalrightmargin, so:
     \hskip-\linewidth \hskip-\@totalleftmargin \hskip\columnwidth}%
 \MakeFramed {\advance\hsize-\width
   \@totalleftmargin\z@ \linewidth\hsize
   \@setminipage}}%
 {\par\unskip\endMakeFramed%
 \at@end@of@kframe}
\makeatother

\definecolor{shadecolor}{rgb}{.97, .97, .97}
\definecolor{messagecolor}{rgb}{0, 0, 0}
\definecolor{warningcolor}{rgb}{1, 0, 1}
\definecolor{errorcolor}{rgb}{1, 0, 0}
\newenvironment{knitrout}{}{} % an empty environment to be redefined in TeX

\usepackage{alltt}
\usepackage{graphicx}
\usepackage{tikz}
\setbeameroption{hide notes}
\setbeamertemplate{note page}[plain]
\usepackage{listings}

% get rid of junk
\usetheme{default}
\usefonttheme[onlymath]{serif}
\beamertemplatenavigationsymbolsempty
\hypersetup{pdfpagemode=UseNone} % don't show bookmarks on initial view

% named colors
\definecolor{offwhite}{RGB}{255,250,240}
\definecolor{gray}{RGB}{155,155,155}

\ifx\notescolors\undefined % slides

  \definecolor{foreground}{RGB}{80,80,80}
  \definecolor{background}{RGB}{255,255,255}
  \definecolor{title}{RGB}{255,199,0}
  \definecolor{subtitle}{RGB}{89,132,212}
  \definecolor{hilit}{RGB}{248,117,79}
  \definecolor{vhilit}{RGB}{255,111,207}
  \definecolor{lolit}{RGB}{200,200,200}
  \definecolor{lit}{RGB}{255,199,0}
  \definecolor{mdlit}{RGB}{89,132,212}
  \definecolor{link}{RGB}{248,117,79}

\else % notes
  \definecolor{background}{RGB}{255,255,255}
  \definecolor{foreground}{RGB}{24,24,24}
  \definecolor{title}{RGB}{27,94,134}
  \definecolor{subtitle}{RGB}{22,175,124}
  \definecolor{hilit}{RGB}{122,0,128}
  \definecolor{vhilit}{RGB}{255,0,128}
  \definecolor{lolit}{RGB}{95,95,95}
\fi
\definecolor{nhilit}{RGB}{128,0,128}  % hilit color in notes
\definecolor{nvhilit}{RGB}{255,0,128} % vhilit for notes

\newcommand{\hilit}{\color{hilit}}
\newcommand{\vhilit}{\color{vhilit}}
\newcommand{\nhilit}{\color{nhilit}}
\newcommand{\nvhilit}{\color{nvhilit}}
\newcommand{\lit}{\color{lit}}
\newcommand{\mdlit}{\color{mdlit}}
\newcommand{\lolit}{\color{lolit}}

% use those colors
\setbeamercolor{titlelike}{fg=title}
\setbeamercolor{subtitle}{fg=subtitle}
\setbeamercolor{frametitle}{fg=gray}
\setbeamercolor{structure}{fg=subtitle}
\setbeamercolor{institute}{fg=lolit}
\setbeamercolor{normal text}{fg=foreground,bg=background}
%\setbeamercolor{item}{fg=foreground} % color of bullets
%\setbeamercolor{subitem}{fg=hilit}
%\setbeamercolor{itemize/enumerate subbody}{fg=lolit}
\setbeamertemplate{itemize subitem}{{\textendash}}
\setbeamerfont{itemize/enumerate subbody}{size=\footnotesize}
\setbeamerfont{itemize/enumerate subitem}{size=\footnotesize}

% center title of slides
\setbeamertemplate{blocks}[rounded]
\setbeamertemplate{frametitle}[default][center]
% margins
\setbeamersize{text margin left=25pt,text margin right=25pt}

% page number
\setbeamertemplate{footline}{%
    \raisebox{5pt}{\makebox[\paperwidth]{\hfill\makebox[20pt]{\lolit
          \scriptsize\insertframenumber}}}\hspace*{5pt}}

% add a bit of space at the top of the notes page
\addtobeamertemplate{note page}{\setlength{\parskip}{12pt}}

% default link color
\hypersetup{colorlinks, urlcolor={link}}

\ifx\notescolors\undefined % slides
  % set up listing environment
  \lstset{language=bash,
          basicstyle=\ttfamily\scriptsize,
          frame=single,
          commentstyle=,
          backgroundcolor=\color{darkgray},
          showspaces=false,
          showstringspaces=false
          }
\else % notes
  \lstset{language=bash,
          basicstyle=\ttfamily\scriptsize,
          frame=single,
          commentstyle=,
          backgroundcolor=\color{offwhite},
          showspaces=false,
          showstringspaces=false
          }
\fi

% a few macros
\newcommand{\code}[1]{\texttt{#1}}
\newcommand{\hicode}[1]{{\hilit \texttt{#1}}}
\newcommand{\bb}[1]{\begin{block}{#1}}
\newcommand{\eb}{\end{block}}
\newcommand{\bi}{\begin{itemize}}
%\newcommand{\bbi}{\vspace{24pt} \begin{itemize} \itemsep8pt}
\newcommand{\bbi}{\vspace{4pt} \begin{itemize} \itemsep8pt}
\newcommand{\ei}{\end{itemize}}
\newcommand{\bv}{\begin{verbatim}}
\newcommand{\ev}{\end{verbatim}}
\newcommand{\ig}{\includegraphics}
\newcommand{\subt}[1]{{\footnotesize \color{subtitle} {#1}}}
\newcommand{\ttsm}{\tt \small}
\newcommand{\ttfn}{\tt \footnotesize}
\newcommand{\figh}[2]{\centerline{\includegraphics[height=#2\textheight]{#1}}}
\newcommand{\figw}[2]{\centerline{\includegraphics[width=#2\textwidth]{#1}}}



%------------------------------------------------
% end of header
%------------------------------------------------

\title{Advanced Loops}
\subtitle{STAT 133}
\author{\href{http://www.gastonsanchez.com}{Gaston Sanchez}}
\institute{Department of Statistics, UC{\textendash}Berkeley}
\date{\href{http://www.gastonsanchez.com}{\tt \scriptsize \color{foreground} gastonsanchez.com}
\\[-4pt]
\href{http://github.com/gastonstat/stat133}{\tt \scriptsize \color{foreground} github.com/gastonstat/stat133}
\\[-4pt]
{\scriptsize Course web: \href{http://www.gastonsanchez.com/stat133}{\tt gastonsanchez.com/stat133}}
}
\IfFileExists{upquote.sty}{\usepackage{upquote}}{}
\begin{document}


{
  \setbeamertemplate{footline}{} % no page number here
  \frame{
    \titlepage
  } 
}

%------------------------------------------------

\begin{frame}
\begin{center}
\Huge{\hilit{Advanced Looping}}
\end{center}
\end{frame}

%------------------------------------------------

\begin{frame}[fragile]
\frametitle{Outline}

\bbi
  \item Vectorizing a function
  \item Loops over elements of data structures
\ei

\end{frame}

%------------------------------------------------

\begin{frame}[fragile]
\frametitle{Motivation}

\begin{knitrout}\footnotesize
\definecolor{shadecolor}{rgb}{0.969, 0.969, 0.969}\color{fgcolor}\begin{kframe}
\begin{alltt}
\hlcom{# fahrenheit to celsius}
\hlstd{to_celsius} \hlkwb{<-} \hlkwa{function}\hlstd{(}\hlkwc{x}\hlstd{) \{}
  \hlstd{(x} \hlopt{-} \hlnum{32}\hlstd{)} \hlopt{*} \hlstd{(}\hlnum{5}\hlopt{/}\hlnum{9}\hlstd{)}
\hlstd{\}}
\end{alltt}
\end{kframe}
\end{knitrout}

The function \code{to\_celsius()} happens to be a vectorized function:
\begin{knitrout}\footnotesize
\definecolor{shadecolor}{rgb}{0.969, 0.969, 0.969}\color{fgcolor}\begin{kframe}
\begin{alltt}
\hlkwd{to_celsius}\hlstd{(}\hlkwd{c}\hlstd{(}\hlnum{32}\hlstd{,} \hlnum{40}\hlstd{,} \hlnum{50}\hlstd{,} \hlnum{60}\hlstd{,} \hlnum{70}\hlstd{))}
\end{alltt}
\begin{verbatim}
## [1]  0.000000  4.444444 10.000000 15.555556 21.111111
\end{verbatim}
\end{kframe}
\end{knitrout}

\end{frame}

%------------------------------------------------

\begin{frame}[fragile]
\frametitle{Motivation}

\bi
  \item In general, R functions defined on scalar values are expected to be vectorized
  \item You should have noticed that many functions in R are vectorized
\ei

\end{frame}

%------------------------------------------------

\begin{frame}[fragile]
\frametitle{Motivation}

What happens in this situation?
\begin{knitrout}\footnotesize
\definecolor{shadecolor}{rgb}{0.969, 0.969, 0.969}\color{fgcolor}\begin{kframe}
\begin{alltt}
\hlcom{# trying to_celsius() on a list}
\hlkwd{to_celsius}\hlstd{(}\hlkwd{list}\hlstd{(}\hlnum{32}\hlstd{,} \hlnum{40}\hlstd{,} \hlnum{50}\hlstd{,} \hlnum{60}\hlstd{,} \hlnum{70}\hlstd{))}
\end{alltt}
\end{kframe}
\end{knitrout}

\end{frame}

%------------------------------------------------

\begin{frame}[fragile]
\frametitle{Motivation}

\begin{knitrout}\footnotesize
\definecolor{shadecolor}{rgb}{0.969, 0.969, 0.969}\color{fgcolor}\begin{kframe}
\begin{alltt}
\hlcom{# trying to_celsius() on a list}
\hlkwd{to_celsius}\hlstd{(}\hlkwd{list}\hlstd{(}\hlnum{32}\hlstd{,} \hlnum{40}\hlstd{,} \hlnum{50}\hlstd{,} \hlnum{60}\hlstd{,} \hlnum{70}\hlstd{))}
\end{alltt}


{\ttfamily\noindent\bfseries\color{errorcolor}{\#\# Error in x - 32: non-numeric argument to binary operator}}\end{kframe}
\end{knitrout}

\code{to\_celsius()} does not work with a list

\end{frame}

%------------------------------------------------

\begin{frame}[fragile]
\frametitle{Motivation}

One solution is to use a {\hilit \code{for}} loop:
\begin{knitrout}\footnotesize
\definecolor{shadecolor}{rgb}{0.969, 0.969, 0.969}\color{fgcolor}\begin{kframe}
\begin{alltt}
\hlstd{temps_farhenheit} \hlkwb{<-} \hlkwd{list}\hlstd{(}\hlnum{32}\hlstd{,} \hlnum{40}\hlstd{,} \hlnum{50}\hlstd{,} \hlnum{60}\hlstd{,} \hlnum{70}\hlstd{)}

\hlstd{temps_celsius} \hlkwb{<-} \hlkwd{numeric}\hlstd{(}\hlnum{5}\hlstd{)}
\hlkwa{for} \hlstd{(i} \hlkwa{in} \hlnum{1}\hlopt{:}\hlnum{5}\hlstd{) \{}
  \hlstd{temps_celsius[i]} \hlkwb{<-} \hlkwd{to_celsius}\hlstd{(temps_farhenheit[[i]])}
\hlstd{\}}

\hlstd{temps_celsius}
\end{alltt}
\begin{verbatim}
## [1]  0.000000  4.444444 10.000000 15.555556 21.111111
\end{verbatim}
\end{kframe}
\end{knitrout}

\end{frame}

%------------------------------------------------

\begin{frame}[fragile]
\frametitle{Vectorizing Functions - Vectors}

\bbi
  \item R provides a set of functions to ``vectorize'' functions over the elements of data structures:
  \bi
    \item \code{lapply()}, \code{sapply()}, \code{apply()}, etc
  \ei
  \item These functions allow us to avoid writing loops
  \item These are functions that have grown organically
  \item They have common names but unfortunately not all of them use the same arguments naming conventions
\ei

\end{frame}

%------------------------------------------------

\begin{frame}
\begin{center}
\Huge{\hilit{\code{lapply()}}}
\end{center}
\end{frame}

%------------------------------------------------

\begin{frame}
\frametitle{Loops over vectors or lists}

\bi
  \item The simplest apply function is {\hilit \code{lapply()}}
  \item \code{lapply()} stands for \textbf{list apply}
  \item It takes a list or vector and a function as inputs
  \item It applies the function to each element of the list
  \item The output is another list
\ei

\end{frame}

%------------------------------------------------

\begin{frame}[fragile]
\frametitle{\code{lapply()}}

\begin{knitrout}\footnotesize
\definecolor{shadecolor}{rgb}{0.969, 0.969, 0.969}\color{fgcolor}\begin{kframe}
\begin{alltt}
\hlstd{players} \hlkwb{<-} \hlkwd{list}\hlstd{(}
  \hlkwc{warriors} \hlstd{=} \hlkwd{c}\hlstd{(}\hlstr{'kurry'}\hlstd{,} \hlstr{'iguodala'}\hlstd{,} \hlstr{'thompson'}\hlstd{,} \hlstr{'green'}\hlstd{),}
  \hlkwc{cavaliers} \hlstd{=} \hlkwd{c}\hlstd{(}\hlstr{'james'}\hlstd{,} \hlstr{'shumpert'}\hlstd{,} \hlstr{'thompson'}\hlstd{),}
  \hlkwc{rockets} \hlstd{=} \hlkwd{c}\hlstd{(}\hlstr{'harden'}\hlstd{,} \hlstr{'howard'}\hlstd{)}
\hlstd{)}

\hlkwd{lapply}\hlstd{(players, length)}
\end{alltt}
\begin{verbatim}
## $warriors
## [1] 4
## 
## $cavaliers
## [1] 3
## 
## $rockets
## [1] 2
\end{verbatim}
\end{kframe}
\end{knitrout}

\end{frame}

%------------------------------------------------

\begin{frame}[fragile]
\frametitle{\code{lapply()}}

\begin{knitrout}\footnotesize
\definecolor{shadecolor}{rgb}{0.969, 0.969, 0.969}\color{fgcolor}\begin{kframe}
\begin{alltt}
\hlcom{# convert to upper case}
\hlkwd{lapply}\hlstd{(players, toupper)}
\end{alltt}
\begin{verbatim}
## $warriors
## [1] "KURRY"    "IGUODALA" "THOMPSON" "GREEN"   
## 
## $cavaliers
## [1] "JAMES"    "SHUMPERT" "THOMPSON"
## 
## $rockets
## [1] "HARDEN" "HOWARD"
\end{verbatim}
\end{kframe}
\end{knitrout}

\end{frame}

%------------------------------------------------

\begin{frame}[fragile]
\frametitle{\code{lapply()}}

You can pass arguments to the applied functions
\begin{knitrout}\footnotesize
\definecolor{shadecolor}{rgb}{0.969, 0.969, 0.969}\color{fgcolor}\begin{kframe}
\begin{alltt}
\hlcom{# collapsing with paste()}
\hlkwd{lapply}\hlstd{(players, paste,} \hlkwc{collapse} \hlstd{=} \hlstr{'-'}\hlstd{)}
\end{alltt}
\begin{verbatim}
## $warriors
## [1] "kurry-iguodala-thompson-green"
## 
## $cavaliers
## [1] "james-shumpert-thompson"
## 
## $rockets
## [1] "harden-howard"
\end{verbatim}
\end{kframe}
\end{knitrout}

\end{frame}

%------------------------------------------------

\begin{frame}[fragile]
\frametitle{\code{lapply()}}

You can pass your own functions
\begin{knitrout}\footnotesize
\definecolor{shadecolor}{rgb}{0.969, 0.969, 0.969}\color{fgcolor}\begin{kframe}
\begin{alltt}
\hlstd{num_chars} \hlkwb{<-} \hlkwa{function}\hlstd{(}\hlkwc{x}\hlstd{) \{}
  \hlkwd{nchar}\hlstd{(x)}
\hlstd{\}}

\hlkwd{lapply}\hlstd{(players, num_chars)}
\end{alltt}
\begin{verbatim}
## $warriors
## [1] 5 8 8 5
## 
## $cavaliers
## [1] 5 8 8
## 
## $rockets
## [1] 6 6
\end{verbatim}
\end{kframe}
\end{knitrout}

\end{frame}

%------------------------------------------------

\begin{frame}[fragile]
\frametitle{\code{Anonymous functions}}

You can define a function with no name (i.e. anonymous function):
\begin{knitrout}\footnotesize
\definecolor{shadecolor}{rgb}{0.969, 0.969, 0.969}\color{fgcolor}\begin{kframe}
\begin{alltt}
\hlcom{# anonymous function}
\hlkwd{lapply}\hlstd{(players,} \hlkwa{function}\hlstd{(}\hlkwc{x}\hlstd{)} \hlkwd{paste}\hlstd{(}\hlstr{'mr'}\hlstd{, x))}
\end{alltt}
\begin{verbatim}
## $warriors
## [1] "mr kurry"    "mr iguodala" "mr thompson" "mr green"   
## 
## $cavaliers
## [1] "mr james"    "mr shumpert" "mr thompson"
## 
## $rockets
## [1] "mr harden" "mr howard"
\end{verbatim}
\end{kframe}
\end{knitrout}

\end{frame}

%------------------------------------------------

\begin{frame}[fragile]
\frametitle{\code{Anonymous functions}}

\begin{knitrout}\footnotesize
\definecolor{shadecolor}{rgb}{0.969, 0.969, 0.969}\color{fgcolor}\begin{kframe}
\begin{alltt}
\hlcom{# anonymous function}
\hlkwd{lapply}\hlstd{(players,} \hlkwa{function}\hlstd{(}\hlkwc{x}\hlstd{)} \hlkwd{grep}\hlstd{(}\hlstr{'a'}\hlstd{, x,} \hlkwc{value} \hlstd{=} \hlnum{TRUE}\hlstd{))}
\end{alltt}
\begin{verbatim}
## $warriors
## [1] "iguodala"
## 
## $cavaliers
## [1] "james"
## 
## $rockets
## [1] "harden" "howard"
\end{verbatim}
\end{kframe}
\end{knitrout}

\end{frame}

%------------------------------------------------

\begin{frame}[fragile]
\frametitle{\code{lapply()}}

Remember that a \code{data.frame} is internally stored as a list:
\begin{knitrout}\footnotesize
\definecolor{shadecolor}{rgb}{0.969, 0.969, 0.969}\color{fgcolor}\begin{kframe}
\begin{alltt}
\hlstd{df} \hlkwb{<-} \hlkwd{data.frame}\hlstd{(}
  \hlkwc{name} \hlstd{=} \hlkwd{c}\hlstd{(}\hlstr{'Luke'}\hlstd{,} \hlstr{'Leia'}\hlstd{,} \hlstr{'R2-D2'}\hlstd{,} \hlstr{'C-3PO'}\hlstd{),}
  \hlkwc{gender} \hlstd{=} \hlkwd{c}\hlstd{(}\hlstr{'male'}\hlstd{,} \hlstr{'female'}\hlstd{,} \hlstr{'male'}\hlstd{,} \hlstr{'male'}\hlstd{),}
  \hlkwc{height} \hlstd{=} \hlkwd{c}\hlstd{(}\hlnum{1.72}\hlstd{,} \hlnum{1.50}\hlstd{,} \hlnum{0.96}\hlstd{,} \hlnum{1.67}\hlstd{),}
  \hlkwc{weight} \hlstd{=} \hlkwd{c}\hlstd{(}\hlnum{77}\hlstd{,} \hlnum{49}\hlstd{,} \hlnum{32}\hlstd{,} \hlnum{75}\hlstd{)}
\hlstd{)}
\end{alltt}
\end{kframe}
\end{knitrout}

\end{frame}

%------------------------------------------------

\begin{frame}[fragile]
\frametitle{\code{lapply()}}

Remember that a \code{data.frame} is internally stored as a list:
\begin{knitrout}\footnotesize
\definecolor{shadecolor}{rgb}{0.969, 0.969, 0.969}\color{fgcolor}\begin{kframe}
\begin{alltt}
\hlkwd{lapply}\hlstd{(df, class)}
\end{alltt}
\begin{verbatim}
## $name
## [1] "factor"
## 
## $gender
## [1] "factor"
## 
## $height
## [1] "numeric"
## 
## $weight
## [1] "numeric"
\end{verbatim}
\end{kframe}
\end{knitrout}

\end{frame}

%------------------------------------------------

\begin{frame}
\begin{center}
\Huge{\hilit{\code{sapply()}}}
\end{center}
\end{frame}

%------------------------------------------------

\begin{frame}
\frametitle{Loops over vectors or lists}

\bi
  \item \code{sapply()} is a modified version of {\hilit \code{lapply()}}
  \item \code{sapply()} stands for \textbf{simplified apply}
  \item It takes a list or vector and a function as inputs
  \item It applies the function to each element of the list
  \item \code{sapply()} attempts to simplify the output (possibly as a vector or list)
\ei

\end{frame}

%------------------------------------------------

\begin{frame}[fragile]
\frametitle{\code{sapply()}}

\begin{knitrout}\footnotesize
\definecolor{shadecolor}{rgb}{0.969, 0.969, 0.969}\color{fgcolor}\begin{kframe}
\begin{alltt}
\hlstd{players} \hlkwb{<-} \hlkwd{list}\hlstd{(}
  \hlkwc{warriors} \hlstd{=} \hlkwd{c}\hlstd{(}\hlstr{'kurry'}\hlstd{,} \hlstr{'iguodala'}\hlstd{,} \hlstr{'thompson'}\hlstd{,} \hlstr{'green'}\hlstd{),}
  \hlkwc{cavaliers} \hlstd{=} \hlkwd{c}\hlstd{(}\hlstr{'james'}\hlstd{,} \hlstr{'shumpert'}\hlstd{,} \hlstr{'thompson'}\hlstd{),}
  \hlkwc{rockets} \hlstd{=} \hlkwd{c}\hlstd{(}\hlstr{'harden'}\hlstd{,} \hlstr{'howard'}\hlstd{)}
\hlstd{)}

\hlkwd{sapply}\hlstd{(players, length)}
\end{alltt}
\begin{verbatim}
##  warriors cavaliers   rockets 
##         4         3         2
\end{verbatim}
\end{kframe}
\end{knitrout}

\end{frame}

%------------------------------------------------

\begin{frame}[fragile]
\frametitle{\code{sapply()}}

\begin{knitrout}\footnotesize
\definecolor{shadecolor}{rgb}{0.969, 0.969, 0.969}\color{fgcolor}\begin{kframe}
\begin{alltt}
\hlkwd{sapply}\hlstd{(players, nchar)}
\end{alltt}
\begin{verbatim}
## $warriors
## [1] 5 8 8 5
## 
## $cavaliers
## [1] 5 8 8
## 
## $rockets
## [1] 6 6
\end{verbatim}
\end{kframe}
\end{knitrout}

when the output cannot be simplified, \code{sapply()} returns the same output as \code{lapply()}

\end{frame}

%------------------------------------------------

\begin{frame}
\begin{center}
\Huge{\hilit{\code{apply()}}}
\end{center}
\end{frame}

%------------------------------------------------

\begin{frame}[fragile]
\frametitle{Loops on matrices (or arrays)}

Consider a matrix:
\begin{knitrout}\footnotesize
\definecolor{shadecolor}{rgb}{0.969, 0.969, 0.969}\color{fgcolor}\begin{kframe}
\begin{alltt}
\hlstd{(m} \hlkwb{<-} \hlkwd{matrix}\hlstd{(}\hlnum{1}\hlopt{:}\hlnum{20}\hlstd{,} \hlnum{4}\hlstd{,} \hlnum{5}\hlstd{))}
\end{alltt}
\begin{verbatim}
##      [,1] [,2] [,3] [,4] [,5]
## [1,]    1    5    9   13   17
## [2,]    2    6   10   14   18
## [3,]    3    7   11   15   19
## [4,]    4    8   12   16   20
\end{verbatim}
\end{kframe}
\end{knitrout}

How can we get the median of each row?

\end{frame}

%------------------------------------------------

\begin{frame}[fragile]
\frametitle{Loops on matrices (or arrays)}

We could write something like this (not recommended)
\begin{knitrout}\footnotesize
\definecolor{shadecolor}{rgb}{0.969, 0.969, 0.969}\color{fgcolor}\begin{kframe}
\begin{alltt}
\hlstd{medians} \hlkwb{<-} \hlkwd{numeric}\hlstd{(}\hlkwd{nrow}\hlstd{(m))}

\hlstd{medians[}\hlnum{1}\hlstd{]} \hlkwb{<-} \hlkwd{median}\hlstd{(m[}\hlnum{1}\hlstd{, ])}
\hlstd{medians[}\hlnum{2}\hlstd{]} \hlkwb{<-} \hlkwd{median}\hlstd{(m[}\hlnum{2}\hlstd{, ])}
\hlstd{medians[}\hlnum{3}\hlstd{]} \hlkwb{<-} \hlkwd{median}\hlstd{(m[}\hlnum{3}\hlstd{, ])}
\hlstd{medians[}\hlnum{4}\hlstd{]} \hlkwb{<-} \hlkwd{median}\hlstd{(m[}\hlnum{4}\hlstd{, ])}
\end{alltt}
\end{kframe}
\end{knitrout}

\end{frame}

%------------------------------------------------

\begin{frame}[fragile]
\frametitle{Loops on matrices (or arrays)}

Repetition is error prone:
\begin{knitrout}\footnotesize
\definecolor{shadecolor}{rgb}{0.969, 0.969, 0.969}\color{fgcolor}\begin{kframe}
\begin{alltt}
\hlstd{medians} \hlkwb{<-} \hlkwd{numeric}\hlstd{(}\hlkwd{nrow}\hlstd{(m))}

\hlstd{medians[}\hlnum{1}\hlstd{]} \hlkwb{<-} \hlkwd{median}\hlstd{(m[}\hlnum{1}\hlstd{, ])}
\hlstd{medians[}\hlnum{2}\hlstd{]} \hlkwb{<-} \hlkwd{median}\hlstd{(m[}\hlnum{2}\hlstd{, ])}
\hlstd{medians[}\hlnum{3}\hlstd{]} \hlkwb{<-} \hlkwd{median}\hlstd{(m[}\hlnum{2}\hlstd{, ])}
\hlstd{medians[}\hlnum{4}\hlstd{]} \hlkwb{<-} \hlkwd{median}\hlstd{(m[}\hlnum{4}\hlstd{, ])}
\end{alltt}
\end{kframe}
\end{knitrout}

\end{frame}

%------------------------------------------------

\begin{frame}[fragile]
\frametitle{Loops on matrices (or arrays)}

We could also write a \code{for} loop
\begin{knitrout}\footnotesize
\definecolor{shadecolor}{rgb}{0.969, 0.969, 0.969}\color{fgcolor}\begin{kframe}
\begin{alltt}
\hlstd{medians} \hlkwb{<-} \hlkwd{numeric}\hlstd{(}\hlkwd{nrow}\hlstd{(m))}

\hlkwa{for} \hlstd{(r} \hlkwa{in} \hlnum{1}\hlopt{:}\hlkwd{nrow}\hlstd{(m)) \{}
  \hlstd{medians[r]} \hlkwb{<-} \hlkwd{median}\hlstd{(m[r, ])}
\hlstd{\}}

\hlstd{medians}
\end{alltt}
\begin{verbatim}
## [1]  9 10 11 12
\end{verbatim}
\end{kframe}
\end{knitrout}

Or we could use the {\hilit \code{apply()}} function

\end{frame}

%------------------------------------------------

\begin{frame}
\frametitle{Loops over matrices or arrays}

\bbi
  \item {\hilit \code{apply()}} is perhaps the most popular apply function
  \item It takes a matrix or array, an index and a function as inputs
  \item Additionaly, it can take more arguments
  \item The \code{MARGIN} index gives the subscript which the function will be applied over
  \bi
    \item \code{MARGIN = 1} indicates rows
    \item \code{MARGIN = 2} indicates columns
    \item \code{MARGIN = c(1, 2)} indicates both rows and columns
  \ei
\ei

\end{frame}

%------------------------------------------------

\begin{frame}[fragile]
\frametitle{\code{apply()}}

\begin{knitrout}\footnotesize
\definecolor{shadecolor}{rgb}{0.969, 0.969, 0.969}\color{fgcolor}\begin{kframe}
\begin{alltt}
\hlstd{(m} \hlkwb{<-} \hlkwd{matrix}\hlstd{(}\hlnum{1}\hlopt{:}\hlnum{20}\hlstd{,} \hlnum{4}\hlstd{,} \hlnum{5}\hlstd{))}
\end{alltt}
\begin{verbatim}
##      [,1] [,2] [,3] [,4] [,5]
## [1,]    1    5    9   13   17
## [2,]    2    6   10   14   18
## [3,]    3    7   11   15   19
## [4,]    4    8   12   16   20
\end{verbatim}
\begin{alltt}
\hlcom{# median of rows}
\hlkwd{apply}\hlstd{(m,} \hlnum{1}\hlstd{, median)}
\end{alltt}
\begin{verbatim}
## [1]  9 10 11 12
\end{verbatim}
\end{kframe}
\end{knitrout}

\end{frame}

%------------------------------------------------

\begin{frame}[fragile]
\frametitle{\code{apply()}}

\begin{knitrout}\footnotesize
\definecolor{shadecolor}{rgb}{0.969, 0.969, 0.969}\color{fgcolor}\begin{kframe}
\begin{alltt}
\hlstd{(m} \hlkwb{<-} \hlkwd{matrix}\hlstd{(}\hlnum{1}\hlopt{:}\hlnum{20}\hlstd{,} \hlnum{4}\hlstd{,} \hlnum{5}\hlstd{))}
\end{alltt}
\begin{verbatim}
##      [,1] [,2] [,3] [,4] [,5]
## [1,]    1    5    9   13   17
## [2,]    2    6   10   14   18
## [3,]    3    7   11   15   19
## [4,]    4    8   12   16   20
\end{verbatim}
\begin{alltt}
\hlcom{# median of columns}
\hlkwd{apply}\hlstd{(m,} \hlnum{2}\hlstd{, median)}
\end{alltt}
\begin{verbatim}
## [1]  2.5  6.5 10.5 14.5 18.5
\end{verbatim}
\end{kframe}
\end{knitrout}

\end{frame}

%------------------------------------------------

\begin{frame}[fragile]
\frametitle{\code{apply()}}

\code{apply()} can be used on data frames
\begin{knitrout}\footnotesize
\definecolor{shadecolor}{rgb}{0.969, 0.969, 0.969}\color{fgcolor}\begin{kframe}
\begin{alltt}
\hlcom{# mean height and weight (on columns)}
\hlkwd{apply}\hlstd{(df[ ,}\hlkwd{c}\hlstd{(}\hlstr{'height'}\hlstd{,} \hlstr{'weight'}\hlstd{)],} \hlnum{2}\hlstd{, mean)}
\end{alltt}
\begin{verbatim}
##  height  weight 
##  1.4625 58.2500
\end{verbatim}
\end{kframe}
\end{knitrout}

\end{frame}

%------------------------------------------------

\begin{frame}[fragile]
\frametitle{\code{apply()}}

\code{apply()} can be used on data frames
\begin{knitrout}\footnotesize
\definecolor{shadecolor}{rgb}{0.969, 0.969, 0.969}\color{fgcolor}\begin{kframe}
\begin{alltt}
\hlcom{# product of height and weight (on rows)}
\hlkwd{apply}\hlstd{(df[ ,}\hlkwd{c}\hlstd{(}\hlstr{'height'}\hlstd{,} \hlstr{'weight'}\hlstd{)],} \hlnum{1}\hlstd{, prod)}
\end{alltt}
\begin{verbatim}
## [1] 132.44  73.50  30.72 125.25
\end{verbatim}
\end{kframe}
\end{knitrout}

\end{frame}

%------------------------------------------------

\begin{frame}
\begin{center}
\Huge{\hilit{\code{tapply()}}}
\end{center}
\end{frame}

%------------------------------------------------

\begin{frame}
\frametitle{Loops over vectors split by a factor}

\bbi
  \item \code{tapply()} 
  \item the name does not mean anything
  \item very useful to \textbf{aggregate} data
\ei

\end{frame}

%------------------------------------------------

\begin{frame}[fragile]
\frametitle{\code{tapply()}}

Say you need to obtain average height and weight by gender
\begin{knitrout}\footnotesize
\definecolor{shadecolor}{rgb}{0.969, 0.969, 0.969}\color{fgcolor}\begin{kframe}
\begin{alltt}
\hlstd{df}
\end{alltt}
\begin{verbatim}
##    name gender height weight
## 1  Luke   male   1.72     77
## 2  Leia female   1.50     49
## 3 R2-D2   male   0.96     32
## 4 C-3PO   male   1.67     75
\end{verbatim}
\end{kframe}
\end{knitrout}

\end{frame}

%------------------------------------------------

\begin{frame}[fragile]
\frametitle{Without \code{tapply()}}

\begin{knitrout}\footnotesize
\definecolor{shadecolor}{rgb}{0.969, 0.969, 0.969}\color{fgcolor}\begin{kframe}
\begin{alltt}
\hlcom{# mean height by gender}
\hlkwd{mean}\hlstd{(df}\hlopt{$}\hlstd{height[df}\hlopt{$}\hlstd{gender} \hlopt{==} \hlstr{'female'}\hlstd{])}
\end{alltt}
\begin{verbatim}
## [1] 1.5
\end{verbatim}
\begin{alltt}
\hlkwd{mean}\hlstd{(df}\hlopt{$}\hlstd{height[df}\hlopt{$}\hlstd{gender} \hlopt{==} \hlstr{'male'}\hlstd{])}
\end{alltt}
\begin{verbatim}
## [1] 1.45
\end{verbatim}
\end{kframe}
\end{knitrout}

\end{frame}

%------------------------------------------------

\begin{frame}[fragile]
\frametitle{Without \code{tapply()}}

\begin{knitrout}\footnotesize
\definecolor{shadecolor}{rgb}{0.969, 0.969, 0.969}\color{fgcolor}\begin{kframe}
\begin{alltt}
\hlcom{# mean weight by gender}
\hlkwd{mean}\hlstd{(df}\hlopt{$}\hlstd{weight[df}\hlopt{$}\hlstd{gender} \hlopt{==} \hlstr{'female'}\hlstd{])}
\end{alltt}
\begin{verbatim}
## [1] 49
\end{verbatim}
\begin{alltt}
\hlkwd{mean}\hlstd{(df}\hlopt{$}\hlstd{weight[df}\hlopt{$}\hlstd{gender} \hlopt{==} \hlstr{'male'}\hlstd{])}
\end{alltt}
\begin{verbatim}
## [1] 61.33333
\end{verbatim}
\end{kframe}
\end{knitrout}

\end{frame}

%------------------------------------------------

\begin{frame}[fragile]
\frametitle{Using \code{tapply()}}

\begin{knitrout}\footnotesize
\definecolor{shadecolor}{rgb}{0.969, 0.969, 0.969}\color{fgcolor}\begin{kframe}
\begin{alltt}
\hlcom{# mean height by gender}
\hlkwd{tapply}\hlstd{(df}\hlopt{$}\hlstd{height, df}\hlopt{$}\hlstd{gender, mean)}
\end{alltt}
\begin{verbatim}
## female   male 
##   1.50   1.45
\end{verbatim}
\begin{alltt}
\hlcom{# mean weight by gender}
\hlkwd{tapply}\hlstd{(df}\hlopt{$}\hlstd{weight, df}\hlopt{$}\hlstd{gender, mean)}
\end{alltt}
\begin{verbatim}
##   female     male 
## 49.00000 61.33333
\end{verbatim}
\end{kframe}
\end{knitrout}

\end{frame}

%------------------------------------------------

\begin{frame}
\begin{center}
\Huge{\hilit{\code{mapply()}}}
\end{center}
\end{frame}

%------------------------------------------------

\begin{frame}
\frametitle{Multiple-Input Apply}

\bbi
  \item \code{lapply()} only accepts a single vector or list to loop over
  \item \code{lapply()} does not give you access to the names of the elements
  \item \code{mapply()} solves this issues
\ei

\end{frame}

%------------------------------------------------

\begin{frame}
\frametitle{Multiple-Input Apply}

\bi
  \item {\hilit \code{mapply()}} stands for \textbf{multiple argument list apply}
  \item it lets you pass in as many vectors as you like
  \item the first argument is the function to be applied
  \item the following arguments are vectors
\ei

\end{frame}

%------------------------------------------------

\begin{frame}[fragile]
\frametitle{\code{mapply()}}

\begin{knitrout}\scriptsize
\definecolor{shadecolor}{rgb}{0.969, 0.969, 0.969}\color{fgcolor}\begin{kframe}
\begin{alltt}
\hlcom{# pasting player name and team}
\hlkwd{mapply}\hlstd{(paste, players,} \hlkwd{names}\hlstd{(players))}
\end{alltt}
\begin{verbatim}
## $warriors
## [1] "kurry warriors"    "iguodala warriors" "thompson warriors"
## [4] "green warriors"   
## 
## $cavaliers
## [1] "james cavaliers"    "shumpert cavaliers" "thompson cavaliers"
## 
## $rockets
## [1] "harden rockets" "howard rockets"
\end{verbatim}
\end{kframe}
\end{knitrout}

\end{frame}

%------------------------------------------------

\begin{frame}[fragile]
\frametitle{\code{mapply()}}

How would you generate this list:
\begin{knitrout}\footnotesize
\definecolor{shadecolor}{rgb}{0.969, 0.969, 0.969}\color{fgcolor}\begin{kframe}
\begin{verbatim}
## [[1]]
## [1] 1 1 1 1
## 
## [[2]]
## [1] 2 2 2
## 
## [[3]]
## [1] 3 3
## 
## [[4]]
## [1] 4
\end{verbatim}
\end{kframe}
\end{knitrout}

\end{frame}

%------------------------------------------------

\begin{frame}[fragile]
\frametitle{\code{mapply()}}

\begin{knitrout}\footnotesize
\definecolor{shadecolor}{rgb}{0.969, 0.969, 0.969}\color{fgcolor}\begin{kframe}
\begin{alltt}
\hlstd{lst} \hlkwb{<-} \hlkwd{vector}\hlstd{(}\hlstr{'list'}\hlstd{,} \hlnum{4}\hlstd{)}
\hlkwa{for} \hlstd{(k} \hlkwa{in} \hlnum{1}\hlopt{:}\hlnum{4}\hlstd{) \{}
  \hlstd{lst[[k]]} \hlkwb{<-} \hlkwd{rep}\hlstd{(k,} \hlnum{5}\hlopt{-}\hlstd{k)}
\hlstd{\}}
\hlstd{lst}
\end{alltt}
\begin{verbatim}
## [[1]]
## [1] 1 1 1 1
## 
## [[2]]
## [1] 2 2 2
## 
## [[3]]
## [1] 3 3
## 
## [[4]]
## [1] 4
\end{verbatim}
\end{kframe}
\end{knitrout}

\end{frame}

%------------------------------------------------

\begin{frame}[fragile]
\frametitle{\code{mapply()}}

\begin{knitrout}\footnotesize
\definecolor{shadecolor}{rgb}{0.969, 0.969, 0.969}\color{fgcolor}\begin{kframe}
\begin{alltt}
\hlcom{# multiple input argument}
\hlkwd{mapply}\hlstd{(rep,} \hlnum{1}\hlopt{:}\hlnum{4}\hlstd{,} \hlnum{4}\hlopt{:}\hlnum{1}\hlstd{)}
\end{alltt}
\begin{verbatim}
## [[1]]
## [1] 1 1 1 1
## 
## [[2]]
## [1] 2 2 2
## 
## [[3]]
## [1] 3 3
## 
## [[4]]
## [1] 4
\end{verbatim}
\end{kframe}
\end{knitrout}

\end{frame}

%------------------------------------------------

\begin{frame}
\begin{center}
\Huge{\hilit{\code{apply()} Related Functions}}
\end{center}
\end{frame}

%------------------------------------------------

\begin{frame}
\frametitle{Related Functions}

Some convenient functions (faster than using \code{apply()})
\bi
  \item \code{colMeans()}
  \item \code{rowMeans()}
  \item \code{colSums()}
  \item \code{rowSums()}
\ei

\end{frame}

%------------------------------------------------

\begin{frame}[fragile]
\frametitle{\code{colMeans()}}

\begin{knitrout}\footnotesize
\definecolor{shadecolor}{rgb}{0.969, 0.969, 0.969}\color{fgcolor}\begin{kframe}
\begin{alltt}
\hlcom{# column means of height and weight}
\hlkwd{colMeans}\hlstd{(df[ ,}\hlkwd{c}\hlstd{(}\hlstr{'height'}\hlstd{,} \hlstr{'weight'}\hlstd{)])}
\end{alltt}
\begin{verbatim}
##  height  weight 
##  1.4625 58.2500
\end{verbatim}
\begin{alltt}
\hlcom{# equivalent to:}
\hlkwd{apply}\hlstd{(df[ ,}\hlkwd{c}\hlstd{(}\hlstr{'height'}\hlstd{,} \hlstr{'weight'}\hlstd{)],} \hlnum{2}\hlstd{, mean)}
\end{alltt}
\begin{verbatim}
##  height  weight 
##  1.4625 58.2500
\end{verbatim}
\end{kframe}
\end{knitrout}

\end{frame}

%------------------------------------------------

\begin{frame}[fragile]
\frametitle{\code{rowMeans()}}

\begin{knitrout}\footnotesize
\definecolor{shadecolor}{rgb}{0.969, 0.969, 0.969}\color{fgcolor}\begin{kframe}
\begin{alltt}
\hlcom{# row means of height and weight}
\hlkwd{rowMeans}\hlstd{(df[ ,}\hlkwd{c}\hlstd{(}\hlstr{'height'}\hlstd{,} \hlstr{'weight'}\hlstd{)])}
\end{alltt}
\begin{verbatim}
## [1] 39.360 25.250 16.480 38.335
\end{verbatim}
\begin{alltt}
\hlcom{# equivalent to:}
\hlkwd{apply}\hlstd{(df[ ,}\hlkwd{c}\hlstd{(}\hlstr{'height'}\hlstd{,} \hlstr{'weight'}\hlstd{)],} \hlnum{1}\hlstd{, mean)}
\end{alltt}
\begin{verbatim}
## [1] 39.360 25.250 16.480 38.335
\end{verbatim}
\end{kframe}
\end{knitrout}

\end{frame}

%------------------------------------------------

\begin{frame}[fragile]
\frametitle{\code{rowSums()}}

\begin{knitrout}\footnotesize
\definecolor{shadecolor}{rgb}{0.969, 0.969, 0.969}\color{fgcolor}\begin{kframe}
\begin{alltt}
\hlcom{# row sums of height and weight}
\hlkwd{rowSums}\hlstd{(df[ ,}\hlkwd{c}\hlstd{(}\hlstr{'height'}\hlstd{,} \hlstr{'weight'}\hlstd{)])}
\end{alltt}
\begin{verbatim}
## [1] 78.72 50.50 32.96 76.67
\end{verbatim}
\begin{alltt}
\hlcom{# equivalent to:}
\hlkwd{apply}\hlstd{(df[ ,}\hlkwd{c}\hlstd{(}\hlstr{'height'}\hlstd{,} \hlstr{'weight'}\hlstd{)],} \hlnum{1}\hlstd{, sum)}
\end{alltt}
\begin{verbatim}
## [1] 78.72 50.50 32.96 76.67
\end{verbatim}
\end{kframe}
\end{knitrout}

\end{frame}

%------------------------------------------------

\begin{frame}[fragile]
\frametitle{\code{colSums()}}

\begin{knitrout}\footnotesize
\definecolor{shadecolor}{rgb}{0.969, 0.969, 0.969}\color{fgcolor}\begin{kframe}
\begin{alltt}
\hlcom{# column sums of height and weight}
\hlkwd{colSums}\hlstd{(df[ ,}\hlkwd{c}\hlstd{(}\hlstr{'height'}\hlstd{,} \hlstr{'weight'}\hlstd{)])}
\end{alltt}
\begin{verbatim}
## height weight 
##   5.85 233.00
\end{verbatim}
\begin{alltt}
\hlcom{# equivalent to:}
\hlkwd{apply}\hlstd{(df[ ,}\hlkwd{c}\hlstd{(}\hlstr{'height'}\hlstd{,} \hlstr{'weight'}\hlstd{)],} \hlnum{2}\hlstd{, sum)}
\end{alltt}
\begin{verbatim}
## height weight 
##   5.85 233.00
\end{verbatim}
\end{kframe}
\end{knitrout}

\end{frame}

%------------------------------------------------

\begin{frame}
\begin{center}
\Huge{\hilit{\code{aggregate()}}}
\end{center}
\end{frame}

%------------------------------------------------

\begin{frame}
\frametitle{Apply a function to data subsets}

\bi
  \item \code{aggregate()} can be thought as a generalization of \code{tapply()}
  \item It splits the data into subsets, and applies a function
  \item The subsets must be provided as a list
  \item The output is returned in a ``convenient'' form
\ei

\end{frame}

%------------------------------------------------

\begin{frame}[fragile]
\frametitle{\code{aggregate()}}

\begin{knitrout}\footnotesize
\definecolor{shadecolor}{rgb}{0.969, 0.969, 0.969}\color{fgcolor}\begin{kframe}
\begin{alltt}
\hlstd{df} \hlkwb{<-} \hlkwd{data.frame}\hlstd{(}
  \hlkwc{name} \hlstd{=} \hlkwd{c}\hlstd{(}\hlstr{'Luke'}\hlstd{,} \hlstr{'Leia'}\hlstd{,} \hlstr{'R2-D2'}\hlstd{,} \hlstr{'C-3PO'}\hlstd{),}
  \hlkwc{gender} \hlstd{=} \hlkwd{c}\hlstd{(}\hlstr{'male'}\hlstd{,} \hlstr{'female'}\hlstd{,} \hlstr{'male'}\hlstd{,} \hlstr{'male'}\hlstd{),}
  \hlkwc{species} \hlstd{=} \hlkwd{c}\hlstd{(}\hlstr{'human'}\hlstd{,} \hlstr{'human'}\hlstd{,} \hlstr{'robot'}\hlstd{,} \hlstr{'robot'}\hlstd{),}
  \hlkwc{height} \hlstd{=} \hlkwd{c}\hlstd{(}\hlnum{1.72}\hlstd{,} \hlnum{1.50}\hlstd{,} \hlnum{0.96}\hlstd{,} \hlnum{1.67}\hlstd{),}
  \hlkwc{weight} \hlstd{=} \hlkwd{c}\hlstd{(}\hlnum{77}\hlstd{,} \hlnum{49}\hlstd{,} \hlnum{32}\hlstd{,} \hlnum{75}\hlstd{)}
\hlstd{)}
\end{alltt}
\end{kframe}
\end{knitrout}

\end{frame}

%------------------------------------------------

\begin{frame}[fragile]
\frametitle{\code{aggregate()}}

\begin{knitrout}\footnotesize
\definecolor{shadecolor}{rgb}{0.969, 0.969, 0.969}\color{fgcolor}\begin{kframe}
\begin{alltt}
\hlcom{# mean height and weight by gender}
\hlkwd{aggregate}\hlstd{(df[ ,}\hlkwd{c}\hlstd{(}\hlstr{'height'}\hlstd{,} \hlstr{'weight'}\hlstd{)],}
          \hlkwd{list}\hlstd{(df}\hlopt{$}\hlstd{gender), mean)}
\end{alltt}
\begin{verbatim}
##   Group.1 height   weight
## 1  female   1.50 49.00000
## 2    male   1.45 61.33333
\end{verbatim}
\end{kframe}
\end{knitrout}

\end{frame}

%------------------------------------------------

\begin{frame}[fragile]
\frametitle{\code{aggregate()}}

\begin{knitrout}\footnotesize
\definecolor{shadecolor}{rgb}{0.969, 0.969, 0.969}\color{fgcolor}\begin{kframe}
\begin{alltt}
\hlcom{# mean height and weight by species}
\hlkwd{aggregate}\hlstd{(df[ ,}\hlkwd{c}\hlstd{(}\hlstr{'height'}\hlstd{,} \hlstr{'weight'}\hlstd{)],}
          \hlkwd{list}\hlstd{(df}\hlopt{$}\hlstd{species), mean)}
\end{alltt}
\begin{verbatim}
##   Group.1 height weight
## 1   human  1.610   63.0
## 2   robot  1.315   53.5
\end{verbatim}
\end{kframe}
\end{knitrout}

\end{frame}

%------------------------------------------------

\begin{frame}[fragile]
\frametitle{\code{aggregate()}}

\begin{knitrout}\footnotesize
\definecolor{shadecolor}{rgb}{0.969, 0.969, 0.969}\color{fgcolor}\begin{kframe}
\begin{alltt}
\hlcom{# mean height and weight by gender and species}
\hlkwd{aggregate}\hlstd{(df[ ,}\hlkwd{c}\hlstd{(}\hlstr{'height'}\hlstd{,} \hlstr{'weight'}\hlstd{)],}
          \hlkwd{list}\hlstd{(df}\hlopt{$}\hlstd{gender, df}\hlopt{$}\hlstd{species), mean)}
\end{alltt}
\begin{verbatim}
##   Group.1 Group.2 height weight
## 1  female   human  1.500   49.0
## 2    male   human  1.720   77.0
## 3    male   robot  1.315   53.5
\end{verbatim}
\end{kframe}
\end{knitrout}

\end{frame}

%------------------------------------------------

\begin{frame}
\begin{center}
\Huge{\hilit{\code{sweep()}}}
\end{center}
\end{frame}

%------------------------------------------------

\begin{frame}
\frametitle{Sweep out array summaries}

\bbi
  \item Sometimes we need to sweep out a summary statistic
  \item e.g. removing the mean on each column
  \item \code{sweep()} is specially designed for this
\ei

\end{frame}

%------------------------------------------------

\begin{frame}[fragile]
\frametitle{\code{sweep()} mean}

\begin{knitrout}\footnotesize
\definecolor{shadecolor}{rgb}{0.969, 0.969, 0.969}\color{fgcolor}\begin{kframe}
\begin{alltt}
\hlcom{# mean height and weight}
\hlstd{hw_mean} \hlkwb{<-} \hlkwd{colMeans}\hlstd{(df[ ,}\hlkwd{c}\hlstd{(}\hlstr{'height'}\hlstd{,} \hlstr{'weight'}\hlstd{)])}

\hlcom{# centering height and weight}
\hlkwd{sweep}\hlstd{(df[ ,}\hlkwd{c}\hlstd{(}\hlstr{'height'}\hlstd{,} \hlstr{'weight'}\hlstd{)],} \hlnum{2}\hlstd{, hw_mean)}
\end{alltt}
\begin{verbatim}
##    height weight
## 1  0.2575  18.75
## 2  0.0375  -9.25
## 3 -0.5025 -26.25
## 4  0.2075  16.75
\end{verbatim}
\end{kframe}
\end{knitrout}

\end{frame}

%------------------------------------------------

\begin{frame}[fragile]
\frametitle{\code{sweep()} median}

\begin{knitrout}\footnotesize
\definecolor{shadecolor}{rgb}{0.969, 0.969, 0.969}\color{fgcolor}\begin{kframe}
\begin{alltt}
\hlcom{# mean height and weight}
\hlstd{hw_median} \hlkwb{<-} \hlkwd{apply}\hlstd{(df[ ,}\hlkwd{c}\hlstd{(}\hlstr{'height'}\hlstd{,} \hlstr{'weight'}\hlstd{)],} \hlnum{2}\hlstd{, median)}

\hlcom{# centering height and weight}
\hlkwd{sweep}\hlstd{(df[ ,}\hlkwd{c}\hlstd{(}\hlstr{'height'}\hlstd{,} \hlstr{'weight'}\hlstd{)],} \hlnum{2}\hlstd{, hw_median)}
\end{alltt}
\begin{verbatim}
##   height weight
## 1  0.135     15
## 2 -0.085    -13
## 3 -0.625    -30
## 4  0.085     13
\end{verbatim}
\end{kframe}
\end{knitrout}

\end{frame}

%------------------------------------------------

\begin{frame}
\begin{center}
\Huge{\hilit{R Package \code{"plyr"}}}
\end{center}
\end{frame}

%------------------------------------------------

\begin{frame}
\frametitle{R package \code{"plyr"}}

\bi
  \item \code{"plyr"} provides alternative functions to the apply-family functions in base R
  \item Don't confuse \code{"plyr"} with \code{"dplyr"}
  \item Functions in \code{"plyr"} are better designed, usually faster, and with better names of arguments
  \item Read the paper \textbf{The Split-Apply-Combine Strategy for Data Analysis}
  \item \url{http://www.jstatsoft.org/v40/i01}
\ei

\end{frame}

%------------------------------------------------


\end{document}

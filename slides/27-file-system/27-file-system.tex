\documentclass[12pt]{beamer}\usepackage[]{graphicx}\usepackage[]{color}
%% maxwidth is the original width if it is less than linewidth
%% otherwise use linewidth (to make sure the graphics do not exceed the margin)
\makeatletter
\def\maxwidth{ %
  \ifdim\Gin@nat@width>\linewidth
    \linewidth
  \else
    \Gin@nat@width
  \fi
}
\makeatother

\definecolor{fgcolor}{rgb}{0.345, 0.345, 0.345}
\newcommand{\hlnum}[1]{\textcolor[rgb]{0.686,0.059,0.569}{#1}}%
\newcommand{\hlstr}[1]{\textcolor[rgb]{0.192,0.494,0.8}{#1}}%
\newcommand{\hlcom}[1]{\textcolor[rgb]{0.678,0.584,0.686}{\textit{#1}}}%
\newcommand{\hlopt}[1]{\textcolor[rgb]{0,0,0}{#1}}%
\newcommand{\hlstd}[1]{\textcolor[rgb]{0.345,0.345,0.345}{#1}}%
\newcommand{\hlkwa}[1]{\textcolor[rgb]{0.161,0.373,0.58}{\textbf{#1}}}%
\newcommand{\hlkwb}[1]{\textcolor[rgb]{0.69,0.353,0.396}{#1}}%
\newcommand{\hlkwc}[1]{\textcolor[rgb]{0.333,0.667,0.333}{#1}}%
\newcommand{\hlkwd}[1]{\textcolor[rgb]{0.737,0.353,0.396}{\textbf{#1}}}%

\usepackage{framed}
\makeatletter
\newenvironment{kframe}{%
 \def\at@end@of@kframe{}%
 \ifinner\ifhmode%
  \def\at@end@of@kframe{\end{minipage}}%
  \begin{minipage}{\columnwidth}%
 \fi\fi%
 \def\FrameCommand##1{\hskip\@totalleftmargin \hskip-\fboxsep
 \colorbox{shadecolor}{##1}\hskip-\fboxsep
     % There is no \\@totalrightmargin, so:
     \hskip-\linewidth \hskip-\@totalleftmargin \hskip\columnwidth}%
 \MakeFramed {\advance\hsize-\width
   \@totalleftmargin\z@ \linewidth\hsize
   \@setminipage}}%
 {\par\unskip\endMakeFramed%
 \at@end@of@kframe}
\makeatother

\definecolor{shadecolor}{rgb}{.97, .97, .97}
\definecolor{messagecolor}{rgb}{0, 0, 0}
\definecolor{warningcolor}{rgb}{1, 0, 1}
\definecolor{errorcolor}{rgb}{1, 0, 0}
\newenvironment{knitrout}{}{} % an empty environment to be redefined in TeX

\usepackage{alltt}
\usepackage{graphicx}
\usepackage{tikz}
\setbeameroption{hide notes}
\setbeamertemplate{note page}[plain]
\usepackage{listings}

% get rid of junk
\usetheme{default}
\usefonttheme[onlymath]{serif}
\beamertemplatenavigationsymbolsempty
\hypersetup{pdfpagemode=UseNone} % don't show bookmarks on initial view

% named colors
\definecolor{offwhite}{RGB}{255,250,240}
\definecolor{gray}{RGB}{155,155,155}

\ifx\notescolors\undefined % slides

  \definecolor{foreground}{RGB}{80,80,80}
  \definecolor{background}{RGB}{255,255,255}
  \definecolor{title}{RGB}{255,199,0}
  \definecolor{subtitle}{RGB}{89,132,212}
  \definecolor{hilit}{RGB}{248,117,79}
  \definecolor{vhilit}{RGB}{255,111,207}
  \definecolor{lolit}{RGB}{200,200,200}
  \definecolor{lit}{RGB}{255,199,0}
  \definecolor{mdlit}{RGB}{89,132,212}
  \definecolor{link}{RGB}{248,117,79}

\else % notes
  \definecolor{background}{RGB}{255,255,255}
  \definecolor{foreground}{RGB}{24,24,24}
  \definecolor{title}{RGB}{27,94,134}
  \definecolor{subtitle}{RGB}{22,175,124}
  \definecolor{hilit}{RGB}{122,0,128}
  \definecolor{vhilit}{RGB}{255,0,128}
  \definecolor{lolit}{RGB}{95,95,95}
\fi
\definecolor{nhilit}{RGB}{128,0,128}  % hilit color in notes
\definecolor{nvhilit}{RGB}{255,0,128} % vhilit for notes

\newcommand{\hilit}{\color{hilit}}
\newcommand{\vhilit}{\color{vhilit}}
\newcommand{\nhilit}{\color{nhilit}}
\newcommand{\nvhilit}{\color{nvhilit}}
\newcommand{\lit}{\color{lit}}
\newcommand{\mdlit}{\color{mdlit}}
\newcommand{\lolit}{\color{lolit}}

% use those colors
\setbeamercolor{titlelike}{fg=title}
\setbeamercolor{subtitle}{fg=subtitle}
\setbeamercolor{frametitle}{fg=gray}
\setbeamercolor{structure}{fg=subtitle}
\setbeamercolor{institute}{fg=lolit}
\setbeamercolor{normal text}{fg=foreground,bg=background}
%\setbeamercolor{item}{fg=foreground} % color of bullets
%\setbeamercolor{subitem}{fg=hilit}
%\setbeamercolor{itemize/enumerate subbody}{fg=lolit}
\setbeamertemplate{itemize subitem}{{\textendash}}
\setbeamerfont{itemize/enumerate subbody}{size=\footnotesize}
\setbeamerfont{itemize/enumerate subitem}{size=\footnotesize}

% center title of slides
\setbeamertemplate{blocks}[rounded]
\setbeamertemplate{frametitle}[default][center]
% margins
\setbeamersize{text margin left=25pt,text margin right=25pt}

% page number
\setbeamertemplate{footline}{%
    \raisebox{5pt}{\makebox[\paperwidth]{\hfill\makebox[20pt]{\lolit
          \scriptsize\insertframenumber}}}\hspace*{5pt}}

% add a bit of space at the top of the notes page
\addtobeamertemplate{note page}{\setlength{\parskip}{12pt}}

% default link color
\hypersetup{colorlinks, urlcolor={link}}

\ifx\notescolors\undefined % slides
  % set up listing environment
  \lstset{language=bash,
          basicstyle=\ttfamily\scriptsize,
          frame=single,
          commentstyle=,
          backgroundcolor=\color{darkgray},
          showspaces=false,
          showstringspaces=false
          }
\else % notes
  \lstset{language=bash,
          basicstyle=\ttfamily\scriptsize,
          frame=single,
          commentstyle=,
          backgroundcolor=\color{offwhite},
          showspaces=false,
          showstringspaces=false
          }
\fi

% a few macros
\newcommand{\code}[1]{\texttt{#1}}
\newcommand{\hicode}[1]{{\hilit \texttt{#1}}}
\newcommand{\bb}[1]{\begin{block}{#1}}
\newcommand{\eb}{\end{block}}
\newcommand{\bi}{\begin{itemize}}
%\newcommand{\bbi}{\vspace{24pt} \begin{itemize} \itemsep8pt}
\newcommand{\bbi}{\vspace{4pt} \begin{itemize} \itemsep8pt}
\newcommand{\ei}{\end{itemize}}
\newcommand{\bv}{\begin{verbatim}}
\newcommand{\ev}{\end{verbatim}}
\newcommand{\ig}{\includegraphics}
\newcommand{\subt}[1]{{\footnotesize \color{subtitle} {#1}}}
\newcommand{\ttsm}{\tt \small}
\newcommand{\ttfn}{\tt \footnotesize}
\newcommand{\figh}[2]{\centerline{\includegraphics[height=#2\textheight]{#1}}}
\newcommand{\figw}[2]{\centerline{\includegraphics[width=#2\textwidth]{#1}}}



%------------------------------------------------
% end of header
%------------------------------------------------

\title{File System Tree}
\subtitle{STAT 133}
\author{\href{http://www.gastonsanchez.com}{Gaston Sanchez}}
\institute{Department of Statistics, UC{\textendash}Berkeley}
\date{\href{http://www.gastonsanchez.com}{\tt \scriptsize \color{foreground} gastonsanchez.com}
\\[-4pt]
\href{http://github.com/gastonstat}{\tt \scriptsize \color{foreground} github.com/gastonstat}
\\[-4pt]
{\scriptsize Course web: \href{http://www.gastonsanchez.com/stat133}{\tt gastonsanchez.com/stat133}}
}
\IfFileExists{upquote.sty}{\usepackage{upquote}}{}
\begin{document}


{
  \setbeamertemplate{footline}{} % no page number here
  \frame{
    \titlepage
  } 
}

%------------------------------------------------

\begin{frame}
\begin{center}
\Huge{\hilit{Managing Files}}
\end{center}
\end{frame}

%------------------------------------------------

\begin{frame}
\frametitle{File Management}

File management is crucial for any data analysis project

Common types of files:
\bi
  \item Data files
  \item Code files (e.g. functions)
  \item Analysis files
  \item Presentation and Report files
\ei

Also, many tools such as R, LaTeX, markdown, etc require knowing where files are located in your computer
\end{frame}

%------------------------------------------------

\begin{frame}
\frametitle{File Management}

Good file managment allows you to:
\bi
  \item find things more easily
  \item make changes more easily
  \item benefit from work you've already done
  \item be understood by others
  \item collaborate with others
\ei

\end{frame}

%------------------------------------------------

\begin{frame}
\frametitle{Why File Management?}

Common tasks
\bi
  \item Access and organize your files
  \item Control creation of files
  \item Control deletion of files
  \item Control relocation of files
  \item Control modification of files
\ei

\end{frame}

%------------------------------------------------

\begin{frame}
\frametitle{Organization of Files}

\begin{center}
{\LARGE How does our computer organize files?}
\end{center}

\end{frame}

%------------------------------------------------

\begin{frame}[fragile]
\frametitle{Files and Directories}
\begin{center}
\ig[width=11cm]{images/macfiles.png}
\end{center}
\end{frame}

%------------------------------------------------

\begin{frame}
\frametitle{Organization of Files}

\bi
  \item The computer organizes files within directories
  \item Directories and files follow a tree structure
  \item A tree structure is a hierarchical structure
  \item Hierarchical means that directories are located inside other directories
\ei

\end{frame}

%------------------------------------------------

\begin{frame}[fragile]
\frametitle{Files and Directories}
\begin{center}
\ig[width=8cm]{images/directories.pdf}
\end{center}
\end{frame}

%------------------------------------------------

\begin{frame}
\frametitle{Filesystem}

\bbi
  \item The nested hierarchy of folders and files on your computer is called the \textbf{filesystem}
  \item The filesystem follows a tree-like structure
\ei

\end{frame}

%------------------------------------------------

\begin{frame}
\begin{center}
\Huge{\hilit{Directories}}
\end{center}
\end{frame}

%------------------------------------------------

\begin{frame}[fragile]
\frametitle{Files and Directories}

\bb{There are two special directories in UNIX-like OS:}
\bbi
  \item The top level directory, named \textbf{"/"}, called the \textbf{root directory}
  \item The \textbf{home directory}, named \textbf{\textasciitilde}, which contains all your files
\ei
\eb

\end{frame}

%------------------------------------------------

\begin{frame}[fragile]
\frametitle{Root Directories}

\bi
  \item A root directory is the first level in a disk (such as a hard drive)
  \item It is the root out of which the file tree ``grows''
  \item All other directories are subdirectories of the root directory
  \item On Unix-like system, including Macs, the root directory is denoted by a forward slash: {\hilit \code{/}}
\ei

\end{frame}

%------------------------------------------------

\begin{frame}
\frametitle{Root Directory}

\bi
  \item The root directory is the most includive folder on the system
  \item The root directory serves as the container of all other files and folders
  \item A Unix-based system (e.g. OS X) has a single root directory
  \item Windows users usually have multiple roots (\code{C:, D:}, etc)
\ei

\end{frame}

%------------------------------------------------

\begin{frame}[fragile]
\frametitle{Root Directories on Windows}

\bi
  \item On Windows computers you can have multiple root directories (one for each storage device)
  \item On Windows, root directories are given a drive letter assignment
  \item On Windows, the most common root directory is {\hilit \code{C:\textbackslash}} (denoting the C partition of the hard drive)
\ei

\end{frame}

%------------------------------------------------

\begin{frame}
\frametitle{Home Directory}
\bi
  \item User's personal files are found in the \code{/Users} directory
  \item e.g. mine is \code{/Users/Gaston}
  \item A user directory is the \textbf{home} directory
\ei
\end{frame}

%------------------------------------------------

\begin{frame}[fragile]
\frametitle{Subdirectories and Parent Directories}

\bi
  \item We store files in subdirectories of the root directory
  \item Inside these subdirectories may be further subdirectories and so on
  \item A directory containing other directories is referred to as the \textbf{parent directory}
  \item Directories inside other directories are referred to as \textbf{child directories}
\ei

\end{frame}

%------------------------------------------------

\begin{frame}[fragile]
\frametitle{Directories and Subdirectories}
\begin{center}
\ig[width=6cm]{images/treefiledir.pdf}
\end{center}

\bi
  \item {\hilit \code{A}} is a child directory of the root directory
  \item {\hilit \code{A}} is the parent directory of \code{B} and \code{C}
\ei

\end{frame}

%------------------------------------------------

\begin{frame}[fragile]
\frametitle{Working Directory}

\bi
  \item Another special type of directory is the so-called \textbf{working directory}
  \item The working directory is the current directory where you perform any task
  \item If you go to your Desktop, then the \code{Desktop} is your current directory
  \item When you use R, the working directory is the directory where the program automatically looks for files 
\ei

\end{frame}

%------------------------------------------------

\begin{frame}[fragile]
\frametitle{Working Directory}
\begin{center}
\ig[width=6cm]{images/treefiledir.pdf}
\end{center}

If you are standing in {\hilit \code{B}}, then this is your working directory

\end{frame}

%------------------------------------------------

\begin{frame}
\begin{center}
\Huge{\hilit{Paths}}
\end{center}
\end{frame}

%------------------------------------------------

\begin{frame}
\frametitle{Path}
\bbi
  \item Each file and directory has a unique name in the filesystem
  \item Such unique name is called a \textbf{path}
  \item The path is simply the desription of where something is located in the filesystem
\ei
\end{frame}

%------------------------------------------------

\begin{frame}
\frametitle{Filesystem}
\bi
  \item The path is a list of directory names separated by slashes
  \item If the path is for a file, then the last element of the path is the file's name
  \item e.g. \code{/Users/Gaston/Documents/data.txt}
  \item A path can be \textbf{absolute} or \textbf{relative}
\ei
\end{frame}

%------------------------------------------------

\begin{frame}
\frametitle{Paths}
\bi
  \item An \textbf{absolute path} is a complete and unambiguous description of where something is in relation to the \textbf{root}
  \item If a path begins with a slash (i.e. the root), then it's called an absolute path
  \item A \textbf{relative} describes where a folder or file is in relation to another folder (typically the working directory)
  \item If a path does not begin with a slash, then it is a relative path
\ei
\end{frame}

%------------------------------------------------

\begin{frame}
\frametitle{Paths}
\bi
  \item There are two special relative paths: \code{.} and \code{..}
  \item The single period \code{.} refers to the current directory
  \item The two periods means the parent directory, one level above
  \item For instance, if the current directory is \code{/Users/XYZ/abc}, then \code{.} refers to this directory, and \code{..} refers to \code{/Users/XYZ}
\ei
\end{frame}

%------------------------------------------------

\begin{frame}[fragile]
\frametitle{Files and Directories}
\begin{center}
\ig[width=8cm]{images/directories.pdf}
\end{center}
\end{frame}

%------------------------------------------------

\begin{frame}[fragile]
\frametitle{Path Names}

\bb{Full path name}
\bi
  \item path from the top level directory, \code{/}, to the file or directory of interest
  \item For \code{mary} the full pathname is: \code{/Users/mary}
\ei
\eb

\end{frame}

%------------------------------------------------

\begin{frame}[fragile]
\frametitle{Files and Directories}
\begin{center}
\ig[width=9cm]{images/treefiledir.pdf}
\end{center}
\end{frame}

%------------------------------------------------

\begin{frame}[fragile]
\frametitle{Path Names}

\bb{Relative path name}
\bi
  \item path from the current directory to the file or directory of interest
  \item Relative path to \code{D} from \code{A}: {\hilit \code{B/D}}
  \item Equivalently: {\hilit \code{./B/D}}  (. refers to current directory)
\ei
\eb

\end{frame}

%------------------------------------------------

\begin{frame}[fragile]
\frametitle{Relative Path Names}
\begin{center}
\ig[width=6cm]{images/treefiledir.pdf}
\end{center}
Relative path to \code{D} from \code{A}: {\hilit \code{B/D}} \\
Equivalently: {\hilit \code{./B/D}}  (. refers to current directory)
\end{frame}

%------------------------------------------------

\begin{frame}[fragile]
\frametitle{Relative Path Names}
\begin{center}
\ig[width=6cm]{images/treefiledir.pdf}
\end{center}
Relative path to \code{D} from \code{C}: {\hilit \code{../B/D}} \\
(.. refers to parent directory)

\end{frame}

%------------------------------------------------

\begin{frame}[fragile]
\frametitle{Relative Path Names}
\begin{center}
\ig[width=6cm]{images/treefiledir.pdf}
\end{center}
Relative path to \code{x} at the top from within \code{C}?

\end{frame}

%------------------------------------------------

\begin{frame}[fragile]
\frametitle{Relative Path Names}
\begin{center}
\ig[width=5cm]{images/treefiledir.pdf}
\end{center}

Relative path to \code{x} at the top from within \code{C}?
\bi
  \item[a)] \code{../A/x}
  \item[b)] \code{../../x}
  \item[c)] \code{../x}
  \item[d)] \code{/x}
\ei

\end{frame}

%------------------------------------------------

\begin{frame}[fragile]
\frametitle{Relative Path Names}
\begin{center}
\ig[width=5cm]{images/treefiledir.pdf}
\end{center}

Relative path to \code{x} in \code{D} from within \code{C}?
\bi
  \item[a)] \code{../D/x}
  \item[b)] \code{../B/D/x}
  \item[c)] \code{../../A/B/D/x}
  \item[d)] \code{/A/B/D/x}
\ei

\end{frame}

%------------------------------------------------

\begin{frame}
\frametitle{Filesystem}
\bi
  \item Root Directory
  \item Home Directory
  \item Working Directory
  \item Directory Tree
  \item Absolute path names
  \item Relative path names
\ei
\end{frame}

%------------------------------------------------

\begin{frame}
\begin{center}
\Huge{\hilit{File Manipulation \\ Commands in R}}
\end{center}
\end{frame}

%------------------------------------------------

\begin{frame}
\frametitle{R File Management Functions}

{\small
\begin{center}
 \begin{tabular}{l l}
  \hline
   function & description \\
  \hline
  \code{getwd()} & shows the current working directory \\  
  \code{list.files()} & see all the files and subdirectories  \\
   & in the current working directory \\
  \code{setwd()} & sets the current working directory \\
  \code{dir.create()} & create a new directory \\
  \code{file.create()} & create a new blank file \\
  \code{cat()} & create a new file and put text into it, \\
   & or append text to an existing file \\
  \code{file.append()} & attempts to append two files \\
  \code{unlink()} & delete files and directories \\
  \code{file.rename()} & rename a file or move a file \\
  \code{file.copy()} & copy a file to another directory \\
  \code{file.exists()} & check whether a file exists \\
  \hline
 \end{tabular}
\end{center}
}

\end{frame}

%------------------------------------------------

\begin{frame}[fragile]
\frametitle{\code{getwd()}}

{\hilit \code{getwd()}} allows you to find your current working directory
\begin{knitrout}\scriptsize
\definecolor{shadecolor}{rgb}{0.969, 0.969, 0.969}\color{fgcolor}\begin{kframe}
\begin{alltt}
\hlcom{# working directory (for these slides)}
\hlkwd{getwd}\hlstd{()}
\end{alltt}
\begin{verbatim}
## [1] "/Users/gaston/Dropbox/course_stat133/stat133/slides/27-file-system"
\end{verbatim}
\end{kframe}
\end{knitrout}

\end{frame}

%------------------------------------------------

\begin{frame}[fragile]
\frametitle{\code{list.files()}}

{\hilit \code{list.files()}} displays the files and subdirectories of the working directory
\begin{knitrout}\scriptsize
\definecolor{shadecolor}{rgb}{0.969, 0.969, 0.969}\color{fgcolor}\begin{kframe}
\begin{alltt}
\hlcom{# files and directories in my working directory}
\hlstd{wd} \hlkwb{<-} \hlkwd{list.files}\hlstd{()}
\hlkwd{head}\hlstd{(wd)}
\end{alltt}
\begin{verbatim}
## [1] "27-file-system-concordance.tex" "27-file-system.log"            
## [3] "27-file-system.nav"             "27-file-system.pdf"            
## [5] "27-file-system.Rnw"             "27-file-system.snm"
\end{verbatim}
\end{kframe}
\end{knitrout}

\end{frame}

%------------------------------------------------

\begin{frame}[fragile]
\frametitle{\code{list.files()}}

You can also specify a different \code{path}
\begin{knitrout}\footnotesize
\definecolor{shadecolor}{rgb}{0.969, 0.969, 0.969}\color{fgcolor}\begin{kframe}
\begin{alltt}
\hlcom{# contents in the stat133 github repo}
\hlkwd{list.files}\hlstd{(}\hlkwc{path} \hlstd{=} \hlstr{'~/Documents/stat133/stat133'}\hlstd{)}
\end{alltt}
\begin{verbatim}
## character(0)
\end{verbatim}
\end{kframe}
\end{knitrout}

\end{frame}

%------------------------------------------------

\begin{frame}[fragile]
\frametitle{\code{setwd()}}

{\hilit \code{setwd()}} allows you to set a working directory (this is where R will look for files and subdirectories)
\begin{knitrout}\footnotesize
\definecolor{shadecolor}{rgb}{0.969, 0.969, 0.969}\color{fgcolor}\begin{kframe}
\begin{alltt}
\hlcom{# setting a working directory}
\hlkwd{setwd}\hlstd{(}\hlstr{'~/Documents/Consulting'}\hlstd{)}
\end{alltt}
\end{kframe}
\end{knitrout}

\end{frame}

%------------------------------------------------

\begin{frame}[fragile]
\frametitle{\code{setwd()}}

Assuming that there is a subdirectory \code{Data} inside \code{Consulting}, we could read a file like so:
\begin{knitrout}\footnotesize
\definecolor{shadecolor}{rgb}{0.969, 0.969, 0.969}\color{fgcolor}\begin{kframe}
\begin{alltt}
\hlcom{# setting a working directory}
\hlstd{df} \hlkwb{<-} \hlkwd{read.csv}\hlstd{(}\hlstr{'Data/dataset.csv'}\hlstd{)}
\end{alltt}
\end{kframe}
\end{knitrout}

\end{frame}

%------------------------------------------------

\begin{frame}[fragile]
\frametitle{\code{dir.create()}}

{\hilit \code{dir.create()}} allows you to create a new directory
\begin{knitrout}\footnotesize
\definecolor{shadecolor}{rgb}{0.969, 0.969, 0.969}\color{fgcolor}\begin{kframe}
\begin{alltt}
\hlcom{# new directory}
\hlkwd{dir.create}\hlstd{(}\hlstr{'/Users/john/Documents/stat133/HW6'}\hlstd{)}

\hlcom{# new directory (Windows)}
\hlkwd{dir.create}\hlstd{(}\hlstr{'C:\textbackslash{}\textbackslash{}Documents\textbackslash{}\textbackslash{}stat133\textbackslash{}\textbackslash{}HW6'}\hlstd{)}
\end{alltt}
\end{kframe}
\end{knitrout}

\end{frame}

%------------------------------------------------

\begin{frame}[fragile]
\frametitle{\code{file.create()}}

{\hilit \code{file.create()}} allows you to create a new blank file
\begin{knitrout}\footnotesize
\definecolor{shadecolor}{rgb}{0.969, 0.969, 0.969}\color{fgcolor}\begin{kframe}
\begin{alltt}
\hlcom{# new file 'functions.R'}
\hlkwd{file.create}\hlstd{(}\hlstr{'/Users/john/Documents/stat133/HW6/functions.R'}\hlstd{)}

\hlcom{# new file (on Windows)}
\hlkwd{file.create}\hlstd{(}\hlstr{'C:\textbackslash{}\textbackslash{}Documents\textbackslash{}\textbackslash{}stat133\textbackslash{}\textbackslash{}HW6\textbackslash{}\textbackslash{}functions.R'}\hlstd{)}
\end{alltt}
\end{kframe}
\end{knitrout}

\end{frame}

%------------------------------------------------

\begin{frame}[fragile]
\frametitle{\code{cat()}}

{\hilit \code{cat()}} can be used to create a new file and put text into it
\begin{knitrout}\footnotesize
\definecolor{shadecolor}{rgb}{0.969, 0.969, 0.969}\color{fgcolor}\begin{kframe}
\begin{alltt}
\hlcom{# new file 'functions.R'}
\hlkwd{cat}\hlstd{(}\hlstr{'# Homework 6'}\hlstd{,}
    \hlstr{'\textbackslash{}n# Your name'}\hlstd{,}
    \hlstr{'\textbackslash{}n# Description'}\hlstd{,}
    \hlkwc{file} \hlstd{=} \hlstr{'/Users/john/Documents/stat133/HW6/myscript.R'}\hlstd{)}
\end{alltt}
\end{kframe}
\end{knitrout}

\end{frame}

%------------------------------------------------

\begin{frame}[fragile]
\frametitle{\code{file.append()}}

{\hilit \code{file.append()}} attempts to append two files
\begin{knitrout}\footnotesize
\definecolor{shadecolor}{rgb}{0.969, 0.969, 0.969}\color{fgcolor}\begin{kframe}
\begin{alltt}
\hlcom{# append two files}
\hlkwd{file.append}\hlstd{(}\hlstr{'data1.csv'}\hlstd{,} \hlstr{'data2.csv'}\hlstd{)}
\end{alltt}
\end{kframe}
\end{knitrout}

\end{frame}

%------------------------------------------------

\begin{frame}[fragile]
\frametitle{\code{unlink()}}

{\hilit \code{unlink()}} deletes files and directories (warning: deletion is permanently)
\begin{knitrout}\footnotesize
\definecolor{shadecolor}{rgb}{0.969, 0.969, 0.969}\color{fgcolor}\begin{kframe}
\begin{alltt}
\hlcom{# delete a file}
\hlkwd{unlink}\hlstd{(}\hlstr{'/Users/john/Documents/stat133/HW6/myscript.R'}\hlstd{)}
\end{alltt}
\end{kframe}
\end{knitrout}

\end{frame}

%------------------------------------------------

\begin{frame}[fragile]
\frametitle{\code{file.rename()}}

{\hilit \code{file.rename()}} renames a file
\begin{knitrout}\footnotesize
\definecolor{shadecolor}{rgb}{0.969, 0.969, 0.969}\color{fgcolor}\begin{kframe}
\begin{alltt}
\hlcom{# rename a file}
\hlkwd{file.rename}\hlstd{(}\hlkwc{from} \hlstd{=} \hlstr{'script.R'}\hlstd{,} \hlkwc{to} \hlstd{=} \hlstr{'analysis.R'}\hlstd{)}
\end{alltt}
\end{kframe}
\end{knitrout}

\end{frame}

%------------------------------------------------

\begin{frame}[fragile]
\frametitle{\code{file.rename()}}

{\hilit \code{file.rename()}} can also be used to fully move a file form one directory to another
\begin{knitrout}\footnotesize
\definecolor{shadecolor}{rgb}{0.969, 0.969, 0.969}\color{fgcolor}\begin{kframe}
\begin{alltt}
\hlcom{# move a file}
\hlkwd{file.rename}\hlstd{(}\hlkwc{from} \hlstd{=} \hlstr{'old-project/analysis.R'}\hlstd{,}
            \hlkwc{to} \hlstd{=} \hlstr{'new-project/analysis.R'}\hlstd{)}
\end{alltt}
\end{kframe}
\end{knitrout}

\end{frame}

%------------------------------------------------

\begin{frame}[fragile]
\frametitle{\code{file.copy()}}

{\hilit \code{file.copy()}} copies a file to another directory
\begin{knitrout}\footnotesize
\definecolor{shadecolor}{rgb}{0.969, 0.969, 0.969}\color{fgcolor}\begin{kframe}
\begin{alltt}
\hlcom{# move a file}
\hlkwd{file.copy}\hlstd{(}\hlkwc{from} \hlstd{=} \hlstr{'old-project/analysis.R'}\hlstd{,}
            \hlkwc{to} \hlstd{=} \hlstr{'new-project/analysis.R'}\hlstd{)}
\end{alltt}
\end{kframe}
\end{knitrout}

\end{frame}

%------------------------------------------------

\begin{frame}[fragile]
\frametitle{\code{file.exists()}}

{\hilit \code{file.exists()}} checks whether a file exists
\begin{knitrout}\footnotesize
\definecolor{shadecolor}{rgb}{0.969, 0.969, 0.969}\color{fgcolor}\begin{kframe}
\begin{alltt}
\hlcom{# checking existance of a file}
\hlkwd{file.exists}\hlstd{(}\hlstr{'homework05_instructions.pdf'}\hlstd{)}
\end{alltt}
\end{kframe}
\end{knitrout}

\end{frame}

%------------------------------------------------

\begin{frame}[fragile]
\frametitle{Related functions}

\bi
  \item \code{file.info()}
  \item \code{file.mode()}
  \item \code{file.mtime()}
  \item \code{file.size()}
  \item \code{file.access()}
  \item \code{system()}
\ei

\end{frame}

%------------------------------------------------



\end{document}

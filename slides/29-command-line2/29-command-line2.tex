\documentclass[12pt]{beamer}\usepackage[]{graphicx}\usepackage[]{color}
%% maxwidth is the original width if it is less than linewidth
%% otherwise use linewidth (to make sure the graphics do not exceed the margin)
\makeatletter
\def\maxwidth{ %
  \ifdim\Gin@nat@width>\linewidth
    \linewidth
  \else
    \Gin@nat@width
  \fi
}
\makeatother

\definecolor{fgcolor}{rgb}{0.345, 0.345, 0.345}
\newcommand{\hlnum}[1]{\textcolor[rgb]{0.686,0.059,0.569}{#1}}%
\newcommand{\hlstr}[1]{\textcolor[rgb]{0.192,0.494,0.8}{#1}}%
\newcommand{\hlcom}[1]{\textcolor[rgb]{0.678,0.584,0.686}{\textit{#1}}}%
\newcommand{\hlopt}[1]{\textcolor[rgb]{0,0,0}{#1}}%
\newcommand{\hlstd}[1]{\textcolor[rgb]{0.345,0.345,0.345}{#1}}%
\newcommand{\hlkwa}[1]{\textcolor[rgb]{0.161,0.373,0.58}{\textbf{#1}}}%
\newcommand{\hlkwb}[1]{\textcolor[rgb]{0.69,0.353,0.396}{#1}}%
\newcommand{\hlkwc}[1]{\textcolor[rgb]{0.333,0.667,0.333}{#1}}%
\newcommand{\hlkwd}[1]{\textcolor[rgb]{0.737,0.353,0.396}{\textbf{#1}}}%

\usepackage{framed}
\makeatletter
\newenvironment{kframe}{%
 \def\at@end@of@kframe{}%
 \ifinner\ifhmode%
  \def\at@end@of@kframe{\end{minipage}}%
  \begin{minipage}{\columnwidth}%
 \fi\fi%
 \def\FrameCommand##1{\hskip\@totalleftmargin \hskip-\fboxsep
 \colorbox{shadecolor}{##1}\hskip-\fboxsep
     % There is no \\@totalrightmargin, so:
     \hskip-\linewidth \hskip-\@totalleftmargin \hskip\columnwidth}%
 \MakeFramed {\advance\hsize-\width
   \@totalleftmargin\z@ \linewidth\hsize
   \@setminipage}}%
 {\par\unskip\endMakeFramed%
 \at@end@of@kframe}
\makeatother

\definecolor{shadecolor}{rgb}{.97, .97, .97}
\definecolor{messagecolor}{rgb}{0, 0, 0}
\definecolor{warningcolor}{rgb}{1, 0, 1}
\definecolor{errorcolor}{rgb}{1, 0, 0}
\newenvironment{knitrout}{}{} % an empty environment to be redefined in TeX

\usepackage{alltt}
\usepackage{graphicx}
\usepackage{tikz}
\setbeameroption{hide notes}
\setbeamertemplate{note page}[plain]
\usepackage{listings}

% get rid of junk
\usetheme{default}
\usefonttheme[onlymath]{serif}
\beamertemplatenavigationsymbolsempty
\hypersetup{pdfpagemode=UseNone} % don't show bookmarks on initial view

% named colors
\definecolor{offwhite}{RGB}{255,250,240}
\definecolor{gray}{RGB}{155,155,155}

\ifx\notescolors\undefined % slides

  \definecolor{foreground}{RGB}{80,80,80}
  \definecolor{background}{RGB}{255,255,255}
  \definecolor{title}{RGB}{255,199,0}
  \definecolor{subtitle}{RGB}{89,132,212}
  \definecolor{hilit}{RGB}{248,117,79}
  \definecolor{vhilit}{RGB}{255,111,207}
  \definecolor{lolit}{RGB}{200,200,200}
  \definecolor{lit}{RGB}{255,199,0}
  \definecolor{mdlit}{RGB}{89,132,212}
  \definecolor{link}{RGB}{248,117,79}

\else % notes
  \definecolor{background}{RGB}{255,255,255}
  \definecolor{foreground}{RGB}{24,24,24}
  \definecolor{title}{RGB}{27,94,134}
  \definecolor{subtitle}{RGB}{22,175,124}
  \definecolor{hilit}{RGB}{122,0,128}
  \definecolor{vhilit}{RGB}{255,0,128}
  \definecolor{lolit}{RGB}{95,95,95}
\fi
\definecolor{nhilit}{RGB}{128,0,128}  % hilit color in notes
\definecolor{nvhilit}{RGB}{255,0,128} % vhilit for notes

\newcommand{\hilit}{\color{hilit}}
\newcommand{\vhilit}{\color{vhilit}}
\newcommand{\nhilit}{\color{nhilit}}
\newcommand{\nvhilit}{\color{nvhilit}}
\newcommand{\lit}{\color{lit}}
\newcommand{\mdlit}{\color{mdlit}}
\newcommand{\lolit}{\color{lolit}}

% use those colors
\setbeamercolor{titlelike}{fg=title}
\setbeamercolor{subtitle}{fg=subtitle}
\setbeamercolor{frametitle}{fg=gray}
\setbeamercolor{structure}{fg=subtitle}
\setbeamercolor{institute}{fg=lolit}
\setbeamercolor{normal text}{fg=foreground,bg=background}
%\setbeamercolor{item}{fg=foreground} % color of bullets
%\setbeamercolor{subitem}{fg=hilit}
%\setbeamercolor{itemize/enumerate subbody}{fg=lolit}
\setbeamertemplate{itemize subitem}{{\textendash}}
\setbeamerfont{itemize/enumerate subbody}{size=\footnotesize}
\setbeamerfont{itemize/enumerate subitem}{size=\footnotesize}

% center title of slides
\setbeamertemplate{blocks}[rounded]
\setbeamertemplate{frametitle}[default][center]
% margins
\setbeamersize{text margin left=25pt,text margin right=25pt}

% page number
\setbeamertemplate{footline}{%
    \raisebox{5pt}{\makebox[\paperwidth]{\hfill\makebox[20pt]{\lolit
          \scriptsize\insertframenumber}}}\hspace*{5pt}}

% add a bit of space at the top of the notes page
\addtobeamertemplate{note page}{\setlength{\parskip}{12pt}}

% default link color
\hypersetup{colorlinks, urlcolor={link}}

\ifx\notescolors\undefined % slides
  % set up listing environment
  \lstset{language=bash,
          basicstyle=\ttfamily\scriptsize,
          frame=single,
          commentstyle=,
          backgroundcolor=\color{darkgray},
          showspaces=false,
          showstringspaces=false
          }
\else % notes
  \lstset{language=bash,
          basicstyle=\ttfamily\scriptsize,
          frame=single,
          commentstyle=,
          backgroundcolor=\color{offwhite},
          showspaces=false,
          showstringspaces=false
          }
\fi

% a few macros
\newcommand{\code}[1]{\texttt{#1}}
\newcommand{\hicode}[1]{{\hilit \texttt{#1}}}
\newcommand{\bb}[1]{\begin{block}{#1}}
\newcommand{\eb}{\end{block}}
\newcommand{\bi}{\begin{itemize}}
%\newcommand{\bbi}{\vspace{24pt} \begin{itemize} \itemsep8pt}
\newcommand{\bbi}{\vspace{4pt} \begin{itemize} \itemsep8pt}
\newcommand{\ei}{\end{itemize}}
\newcommand{\bv}{\begin{verbatim}}
\newcommand{\ev}{\end{verbatim}}
\newcommand{\ig}{\includegraphics}
\newcommand{\subt}[1]{{\footnotesize \color{subtitle} {#1}}}
\newcommand{\ttsm}{\tt \small}
\newcommand{\ttfn}{\tt \footnotesize}
\newcommand{\figh}[2]{\centerline{\includegraphics[height=#2\textheight]{#1}}}
\newcommand{\figw}[2]{\centerline{\includegraphics[width=#2\textwidth]{#1}}}



%------------------------------------------------
% end of header
%------------------------------------------------

\title{Command Line - Part 2}
\subtitle{STAT 133}
\author{\href{http://www.gastonsanchez.com}{Gaston Sanchez}}
\institute{Department of Statistics, UC{\textendash}Berkeley}
\date{\href{http://www.gastonsanchez.com}{\tt \scriptsize \color{foreground} gastonsanchez.com}
\\[-4pt]
\href{http://github.com/gastonstat}{\tt \scriptsize \color{foreground} github.com/gastonstat}
\\[-4pt]
{\scriptsize Course web: \href{http://www.gastonsanchez.com/stat133}{\tt gastonsanchez.com/stat133}}
}
\IfFileExists{upquote.sty}{\usepackage{upquote}}{}
\begin{document}


{
  \setbeamertemplate{footline}{} % no page number here
  \frame{
    \titlepage
  } 
}

%------------------------------------------------

\begin{frame}
\begin{center}
\Huge{\hilit{Standard Input and Output}}
\end{center}
\end{frame}

%------------------------------------------------

\begin{frame}
\frametitle{}

\begin{center}
\Large{Many commands accept {\hilit input} \\ 
and produce {\mdlit output}}
\end{center}

\end{frame}

%------------------------------------------------

\begin{frame}
\frametitle{Input}

Input can come from:
\bbi
  \item the keyboard (a.k.a. \textbf{standard input})
  \item other files
  \item other commands
\ei

\end{frame}

%------------------------------------------------

\begin{frame}
\frametitle{Output}

Output can be:
\bbi
  \item printed on screen 
  \bi
    \item the command's results (a.k.a. \textbf{standard output})
    \item the status and error messages (a.k.a. \textbf{standard error})
  \ei
  \item written to files
  \item sent to other commands
\ei

\end{frame}

%------------------------------------------------

\begin{frame}
\frametitle{Output of commands}

\bi
  \item Consider the command \code{ls}
  \item \code{ls} sends the results to a special file called: \textit{standard output} or \textbf{stdout}
  \item \code{ls} sends status messages to another file called \textit{standard error} or \textbf{stderr}
  \item By default both \textit{stdout} and \textit{stderr} are linked to the screen and not saved into a disk file
\ei

\end{frame}

%------------------------------------------------

\begin{frame}
\frametitle{SI and SO}

\bbi
  \item The ``standard input'' is usually your keyboard
  \item The ``standard output'' is usually your terminal (monitor)
  \item But we can also redirect inputs and outputs
  \item I/O redirection allows us to change where output goes and where input comes from
  \item I/O redirection is done via the {\hilit \code{>}} redirection operator
\ei

\end{frame}

%------------------------------------------------

\begin{frame}
\begin{center}
\Huge{\hilit{Redirection Operator \code{>}}}
\end{center}
\end{frame}

%------------------------------------------------

\begin{frame}[fragile]
\frametitle{The \code{>} operator}

We can tell the shell to send the output of the \code{ls} command to the file \code{ls-output.txt}
\begin{knitrout}\footnotesize
\definecolor{shadecolor}{rgb}{0.969, 0.969, 0.969}\color{fgcolor}\begin{kframe}
\begin{alltt}
ls -l ~/Documents > ls-output.txt
\end{alltt}
\end{kframe}
\end{knitrout}

\end{frame}

%------------------------------------------------

\begin{frame}[fragile]
\frametitle{The \code{>>} operator}

We can tell the shell to send the output of the \code{ls} command and append it to the file \code{ls-output.txt}
\begin{knitrout}\footnotesize
\definecolor{shadecolor}{rgb}{0.969, 0.969, 0.969}\color{fgcolor}\begin{kframe}
\begin{alltt}
ls -l ~/Desktop >> ls-output.txt
\end{alltt}
\end{kframe}
\end{knitrout}

The contents in Desktop are appended to the file \code{ls-output.txt}

\end{frame}

%------------------------------------------------

\begin{frame}
\frametitle{Redirection}

\bbi
  \item[] {\hilit \code{>}} redirects STDOUT to a file
  \item[] {\hilit \code{<}} redirects STDIN from a file
  \item[] {\hilit \code{>>}} redirects STDOUT to a file, but appends rather than overwrites
  \item[] There is also {\hilit \code{<<}} but its use is more advanced than what we'll cover
\ei

\end{frame}

%------------------------------------------------

\begin{frame}
\frametitle{About Redirection}

\bbi
  \item Many times it is useful to send the output of a program to a file rather than to the screen
  \item Redirecting output to files is very common when extracting and combining data (think of merge!)
  \item Think of the redirection operator \code{">"} as an arrow that is pointing to where the output should go
\ei

\end{frame}

%------------------------------------------------

\begin{frame}[fragile]
\frametitle{Joining files with \code{cat}}

We can use {\hilit \code{cat}} and {\hilit \code{>}} to join two or more files:
\begin{knitrout}\footnotesize
\definecolor{shadecolor}{rgb}{0.969, 0.969, 0.969}\color{fgcolor}\begin{kframe}
\begin{alltt}
\hlcom{# remember the files from HW5?}
\hlcom{# (nflweather1960s.csv, ..., nflweather2010s.csv)}
ls nflweather*s.csv

\hlcom{# joining all the decades files in one single file}
cat nflweather*s.csv > allnfl.csv
\end{alltt}
\end{kframe}
\end{knitrout}

{\footnotesize The only issue here is that you would have appended column names}

\end{frame}

%------------------------------------------------

\begin{frame}
\frametitle{Joining files with \code{cat}}

Think about all the steps you would need to join the nfl-weather files without using the command line:
\bi
  \item You would have to open each file
  \item Open a new file \code{allnfl.csv}
  \item Start copy-pasting each adtaset into \code{allnfl.csv}
  \item Close all the decades files
  \item Save and close \code{allnfl.csv}
\ei

\end{frame}

%------------------------------------------------

\begin{frame}
\begin{center}
\Huge{\hilit{Redirection with pipes}}
\end{center}
\end{frame}

%------------------------------------------------

\begin{frame}
\frametitle{Redirection}

\bbi
  \item The idea behind pipes is that rather than redirecting output to a file, we redirect it into another command
  \item STDOUT of one command is used as STDIN to another command
  \item We can redirect inputs and outputs
  \item Redirection is done via the {\hilit \code{|}} pipe operator
\ei

\end{frame}

%------------------------------------------------

\begin{frame}[fragile]
\frametitle{Pipe example}

Let's say you want to count the number of \code{.csv} files in a specfic directory:
\begin{knitrout}\footnotesize
\definecolor{shadecolor}{rgb}{0.969, 0.969, 0.969}\color{fgcolor}\begin{kframe}
\begin{alltt}
\hlcom{# list csv files (one per line)}
\hlstd{ls} \hlopt{-}\hlnum{1} \hlopt{*}\hlstd{.csv}

\hlcom{# piping to count lines with 'wc -l'}
\hlcom{# (how many lines)}
\hlstd{ls} \hlopt{-}\hlnum{1} \hlopt{*}\hlstd{.csv} \hlopt{|} \hlstd{wc} \hlopt{-}\hlstd{l}
\end{alltt}
\end{kframe}
\end{knitrout}

The output of \code{ls -1} is piped to \code{wc -l}

\end{frame}

%------------------------------------------------

\begin{frame}[fragile]
\frametitle{Pipe example}

Let's say you want to inspect the contents of \code{/usr/bin}
\begin{knitrout}\footnotesize
\definecolor{shadecolor}{rgb}{0.969, 0.969, 0.969}\color{fgcolor}\begin{kframe}
\begin{alltt}
\hlcom{# long list of contents}
\hlstd{ls} \hlopt{/}\hlstd{usr}\hlopt{/}\hlstd{bin}

\hlcom{# using 'less' as a pager to see all the contents}
\hlstd{ls} \hlopt{/}\hlstd{usr}\hlopt{/}\hlstd{bin} \hlopt{|} \hlstd{less}
\end{alltt}
\end{kframe}
\end{knitrout}

The output of \code{ls} is piped to \code{less}

\end{frame}

%------------------------------------------------

\begin{frame}
\begin{center}
\Huge{\hilit{Command \code{grep}}}
\end{center}
\end{frame}

%------------------------------------------------

\begin{frame}
\frametitle{Regular Expressions with \code{grep}}

\bbi
  \item We can work with some regular expressions in the command line
  \item For that purpose we use the command {\hilit \code{grep}}
  \item \code{grep} can be very helpful for extracting particular rows from a file
\ei

\end{frame}

%------------------------------------------------

\begin{frame}[fragile]
\frametitle{\code{grep} example}

Consider the data \code{nflweather.csv}
\begin{knitrout}\footnotesize
\definecolor{shadecolor}{rgb}{0.969, 0.969, 0.969}\color{fgcolor}\begin{kframe}
\begin{alltt}
\hlcom{# rows containing Oakland (Raiders)}
grep \hlstr{'Oakland'} nflweather.csv

\hlcom{# rows from 2013}
grep \hlstr{'2013'} nflweather.csv
\end{alltt}
\end{kframe}
\end{knitrout}

\end{frame}

%------------------------------------------------

\begin{frame}[fragile]
\frametitle{\code{grep} example}

Consider the raw data \code{weather\_20131231.csv}
\begin{knitrout}\footnotesize
\definecolor{shadecolor}{rgb}{0.969, 0.969, 0.969}\color{fgcolor}\begin{kframe}
\begin{alltt}
\hlcom{# how many games in 2013}
grep \hlstr{'2013'} weather_20131231.csv | wc -l

\hlcom{# how many games in October 2013}
grep \hlstr{'10/[0-9]*/2013'} weather_20131231.csv | wc -l
\end{alltt}
\end{kframe}
\end{knitrout}

\end{frame}

%------------------------------------------------

\begin{frame}
\begin{center}
\Huge{\hilit{Command \code{curl}}}
\end{center}
\end{frame}

%------------------------------------------------

\begin{frame}
\frametitle{Command \code{curl}}

\bbi
  \item {\hilit \code{curl}} allows you to retrieve content from the Web
  \item \code{curl} stands for ``see URL''
  \item It access Internet files on your behalf, downling the content without any need of a browser window
\ei

\end{frame}

%------------------------------------------------

\begin{frame}[fragile]
\frametitle{\code{curl} example}

\begin{knitrout}\scriptsize
\definecolor{shadecolor}{rgb}{0.969, 0.969, 0.969}\color{fgcolor}\begin{kframe}
\begin{alltt}
\hlcom{# get the content of a URL}
curl \hlstr{"http://www.stat.berkeley.edu/~nolan/data/stat133/Saratoga.txt"}
\end{alltt}
\end{kframe}
\end{knitrout}

\end{frame}

%------------------------------------------------

\begin{frame}[fragile]
\frametitle{\code{curl} example}

\begin{knitrout}\scriptsize
\definecolor{shadecolor}{rgb}{0.969, 0.969, 0.969}\color{fgcolor}\begin{kframe}
\begin{alltt}
\hlcom{# get the content of a URL and save it to a file}
curl \hlstr{"http://www.stat.berkeley.edu/~nolan/data/stat133/Saratoga.txt"} 
-o saratoga.txt


\hlcom{# equivalently}
curl \hlstr{"http://www.stat.berkeley.edu/~nolan/data/stat133/Saratoga.txt"} 
> saratoga.txt
\end{alltt}
\end{kframe}
\end{knitrout}

\end{frame}

%------------------------------------------------

\begin{frame}
\begin{center}
\Huge{\hilit{Overview}}
\end{center}
\end{frame}

%------------------------------------------------

\begin{frame}
\frametitle{What good is it?}

\bbi
  \item Do I really need to learn these commands?
  \item The GUI file finder can do most of what we've seen (e.g. \code{ls}, \code{cd}, \code{mkdir}, \code{rmdir})
  \item Maybe it can't do what \code{cut} can do, but so what?
\ei

\end{frame}

%------------------------------------------------

\begin{frame}
\frametitle{Advantages of shell commands}

\bbi
  \item Shell commands gives us a programatic way to work with files and processes
  \item They allow you to \textbf{record} what you did
  \item They allow you to repeat it another time
  \item Volumne: Have many many operations to perform
  \item Speed: need to perform things quickly
  \item Less error prone: want to reduce mistakes
\ei

\end{frame}

%------------------------------------------------

\begin{frame}
\begin{center}
\Huge{\hilit{Command \code{cut}}}
\end{center}
\end{frame}

%------------------------------------------------

\begin{frame}
\frametitle{Command \code{cut}}

\bbi
  \item {\hilit \code{cut}} is most often used to extract columns of data from a field-delimited file
  \item They allow you to \textbf{record} what you did
  \item They allow you to repeat it another time
\ei

\end{frame}

%------------------------------------------------

\begin{frame}[fragile]
\frametitle{\code{cut} example}

\begin{knitrout}\scriptsize
\definecolor{shadecolor}{rgb}{0.969, 0.969, 0.969}\color{fgcolor}\begin{kframe}
\begin{alltt}
\hlcom{# 2nd column of a tab-separated file}
cut -f 2 starwarstoy.tsv


\hlcom{# 2nd column of a comma-separated file}
cut -f 2 -d \hlstr{","} starwarstoy.csv
\end{alltt}
\end{kframe}
\end{knitrout}

\end{frame}

%------------------------------------------------

\begin{frame}[fragile]
\frametitle{\code{cut} example}

\begin{knitrout}\scriptsize
\definecolor{shadecolor}{rgb}{0.969, 0.969, 0.969}\color{fgcolor}\begin{kframe}
\begin{alltt}
\hlcom{# columns 2-4 of a tab-separated file}
cut -f 2-4 starwarstoy.tsv


\hlcom{# columns 4-6 of a comma-separated file}
cut -f 4-6 -d \hlstr{","} starwarstoy.csv
\end{alltt}
\end{kframe}
\end{knitrout}

\end{frame}

%------------------------------------------------

\begin{frame}[fragile]
\frametitle{\code{cut} example}

\begin{knitrout}\scriptsize
\definecolor{shadecolor}{rgb}{0.969, 0.969, 0.969}\color{fgcolor}\begin{kframe}
\begin{alltt}
\hlcom{# columns 2-3 of first 10 rows in nflweather}
head -n 10 nflweather.csv | cut -f 2-4 
\end{alltt}
\end{kframe}
\end{knitrout}

\end{frame}

%------------------------------------------------



\end{document}
